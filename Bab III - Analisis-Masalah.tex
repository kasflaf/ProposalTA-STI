% ============================================================================================
% BAB III ANALISIS MASALAH
% Pembagian subbab tidak rigid dan dapat bervariasi. Bab ini minimal berisi analisis kebutuhan
% fungsional dan nonfungsional, analisis berbagai alternatif solusi yang dapat ditawarkan, dan
% metode pemilihan solusi yang diusulkan.
% ============================================================================================
\chapter{ANALISIS MASALAH}
\label{chap:analisis-masalah}
\section{Analisis Kondisi Saat Ini}

Seiring dengan meningkatnya kebutuhan akan sistem pelacakan pelari yang akurat dan dapat diakses secara \textit{real-time} dalam penyelenggaraan ITB Ultra-Marathon, penting untuk memahami terlebih dahulu kondisi teknologi dan pendekatan yang saat ini digunakan di industri maupun penelitian. Analisis ini diperlukan untuk mengidentifikasi bagaimana sistem pencatatan waktu konvensional bekerja, sejauh mana teknologi pelacakan kontinu telah berkembang, serta apa saja keterbatasan pada metode prediksi ETA yang tersedia. Pemahaman menyeluruh terhadap kondisi eksisting menjadi dasar dalam merumuskan rancangan \textit{backend} yang mampu menangani skala besar, mendukung prediksi yang lebih kontekstual, dan memenuhi kebutuhan operasional maraton secara spesifik.

\subsection{Model Sistem Berbasis RFID}

Standar industri dalam manajemen lomba lari massal saat ini adalah sistem \textit{timing} berbasis \textit{Radio Frequency Identification} (RFID), khususnya RFID pasif UHF. Model konseptual sistem ini mengandalkan interaksi antara \textit{tag} yang dibawa pelari dengan infrastruktur pembaca yang dipasang pada titik-titik tertentu \cite{RaceID2022}.

Komponen utama dalam sistem ini meliputi:
\begin{enumerate}
    \item Transponder Pasif (Tag): \textit{Chip} tanpa baterai yang ditempelkan pada nomor dada pelari dan hanya aktif ketika menerima energi dari pembaca.
    \item Infrastruktur \textit{Reader/Mats}: Antena atau karpet pembaca yang ditempatkan pada titik diskrit seperti garis \textit{start}, \textit{split} setiap beberapa kilometer, dan garis \textit{finish}.
    \item Dekoder dan Perangkat Lunak: Perangkat yang mengonversi sinyal menjadi ID pelari dan stempel waktu untuk diproses lebih lanjut.
\end{enumerate}

Sistem RFID memiliki keterbatasan mendasar berupa sifat data yang diskrit. Posisi pelari hanya diketahui ketika melewati titik pembaca, sehingga terdapat zona buta antar \textit{checkpoint} \cite{hochreiter2024}. Akibatnya, estimasi posisi maupun kecepatan bergantung pada interpolasi sederhana. Selain itu, pemasangan infrastruktur RFID memerlukan biaya dan tenaga yang signifikan sehingga jumlah titik baca tidak dapat diperbanyak secara fleksibel, terutama pada rute maraton yang panjang.

\subsection{Model Sistem Berbasis LoRaWan dan GPS}

Sebagai perbaikan atas rendahnya resolusi spasial RFID, riset terbaru mengeksplorasi teknologi \textit{Low-Power Wide-Area Network} (LPWAN) seperti LoRaWAN untuk pelacakan kontinu. Riset mengusulkan model di mana pelari membawa pelacak GPS berbasis LoRa yang mengirimkan data posisi secara periodik, tidak terbatas pada titik diskrit\cite{hochreiter2024}.

Model konseptual dalam studi tersebut mencakup:
\begin{enumerate}
    \item \textit{Tracker Node:} Perangkat LoRa berbasis GPS yang mengirimkan pembaruan posisi secara berkala (misalnya setiap 30 detik).
    \item \textit{Mobile Gateway:} \textit{Gateway} bergerak yang dipasang pada kendaraan atau sepeda untuk meningkatkan cakupan sinyal di lingkungan lomba yang dinamis.
    \item \textit{LoRaWAN Network Server (LNS):} Server yang menerima, mendekode, dan menyimpan paket data untuk visualisasi dan analisis lebih lanjut.
\end{enumerate}

Eksperimen pada Vienna City Marathon menunjukkan bahwa pendekatan ini mampu menyediakan data posisi dengan resolusi jauh lebih tinggi dibandingkan RFID, yaitu median interval sekitar 31 detik. Namun, pendekatan ini menuntut infrastruktur \textit{backend} yang sanggup menangani aliran data telemetri yang terus-menerus dan berjumlah besar.

\subsection{Kesenjangan pada Metode Prediksi ETA}

Walaupun pelacakan kontinu semakin memungkinkan, metode prediksi ETA yang digunakan saat ini belum mampu memenuhi kebutuhan konteks lomba lari jarak jauh. Terdapat beberapa kesenjangan utama:

\begin{enumerate}
    \item Model yang Kurang Relevan: Banyak metode ETA yang tersedia dikembangkan untuk kendaraan atau logistik sehingga mengasumsikan kecepatan relatif stabil, bukan perubahan performa fisiologis pelari.
    \item Kurangnya Pertimbangan Konteks Lari: Metode konvensional sering kali tidak memperhitungkan karakteristik seperti elevasi rute, strategi \textit{pacing}, maupun dampak kelelahan.
    \item Kendala Skalabilitas dan Ketersediaan: Sistem ETA eksisting umumnya tidak dirancang untuk menghitung prediksi bagi ribuan peserta secara bersamaan setiap kali data baru diterima. Dalam \textit{event} berskala besar, diperlukan arsitektur \textit{backend} yang mampu menjamin \textit{high availability} dan pemrosesan \textit{real-time}.
\end{enumerate}

Dengan demikian, terdapat kebutuhan akan sistem \textit{backend} yang mampu mengolah data pelacakan kontinu sekaligus menyediakan prediksi ETA yang cerdas, skalabel, dan andal.


\section{Analisis Kebutuhan}
Tahap analisis kebutuhan dilakukan untuk memperoleh gambaran yang komprehensif mengenai tujuan, batasan, serta karakteristik sistem yang akan dibangun. Pada tahap ini, dilakukan identifikasi terhadap permasalahan dan kebutuhan pengguna yang muncul dalam operasional ITB Ultra-Marathon. Selain itu, dirumuskan kebutuhan fungsional yang mendefinisikan kemampuan utama yang harus disediakan sistem, serta kebutuhan nonfungsional yang memastikan sistem mampu beroperasi secara andal, aman, dan skalabel sesuai standar yang dibutuhkan. Hasil dari analisis kebutuhan ini menjadi dasar dalam proses perancangan arsitektur dan implementasi sistem pada tahap berikutnya.

\subsection{Identifikasi Masalah Pengguna}

Berdasarkan observasi dan wawancara mengenai kondisi penyelenggaraan ITB Ultra-Marathon saat ini, terdapat sejumlah permasalahan utama yang dialami oleh panitia, peserta, maupun pihak pendukung acara. Permasalahan tersebut dapat diidentifikasi sebagai berikut.

\begin{enumerate}
    \item Pelacakan posisi pelari masih manual dan tidak real-time.\\
    Panitia dan tim supporter harus meminta pelari melakukan \textit{share location} melalui aplikasi perpesanan untuk mengetahui posisi terkini, sehingga proses pemantauan menjadi tidak efisien.

    \item Mobil penjemput dan \textit{drop bag} sulit untuk tepat waktu karena ETA tidak akurat. \\
    Keterlambatan terjadi karena tidak tersedianya data posisi dan kecepatan pelari yang diperbarui secara kontinu untuk memperkirakan estimasi waktu kedatangan dengan tepat.

    \item Metode ETA yang tersedia (misalnya Google Maps) tidak sesuai untuk konteks pelari. \\
    Pola \textit{pace}, kelelahan, elevasi lintasan, serta karakteristik rute tidak diperhitungkan, sehingga estimasi waktu kedatangan sering meleset dan tidak dapat diandalkan.

    \item Peta digital lomba masih dibuat secara manual. \\
    Peta rute biasanya dibuat menggunakan Google Maps secara manual dan tidak terintegrasi dengan sistem pelacakan, meskipun pelari menggunakan aplikasi tersebut untuk mendapatkan arah rute dan estimasi waktu tempuh.

    \item Tidak ada sistem yang mendukung kebutuhan operasional lomba secara menyeluruh. \\
    Hal-hal seperti manajemen tim, deteksi pergantian pelari untuk kategori relay, pengelolaan pelanggaran dan penalti, pencatatan kehadiran, perkiraan jadwal penjemputan, \textit{drop bag} masih dilakukan secara manual dan rentan terhadap kesalahan.
\end{enumerate}


\subsection{Kebutuhan Fungsional}
Pada tahap ini, disusun kebutuhan fungsional berdasarkan permasalahan yang telah diidentifikasi pada bagian sebelumnya.
Selain itu, kebutuhan ini juga mencerminkan fungsi-fungsi dasar yang harus disediakan oleh sebuah sistem pelacakan marathon modern agar mampu memenuhi kebutuhan panitia, pelari, dan tim supporter secara efektif dan konsisten.

% functional requirement sistem
% \begin{center}
% \begin{longtable}{|c|p{3.5cm}|p{8.5cm}|}
% \caption{Kebutuhan Fungsional Sistem}\label{tab:kebutuhan-fungsional}\\
% \hline
% \textbf{Kode} & \textbf{Kebutuhan} & \textbf{Deskripsi} \\
% \hline
% \endfirsthead

% \hline
% \textbf{Kode} & \textbf{Kebutuhan} & \textbf{Deskripsi} \\
% \hline
% \endhead

% \endfoot

% \hline
% \endlastfoot

% FR-01 & Pendaftaran Peserta &
% Sistem menyediakan proses pendaftaran peserta, termasuk pengisian data identitas dan informasi dasar yang diperlukan untuk mengikuti lomba. \\
% \hline

% FR-02 & Manajemen Tim & Sistem memungkinkan peserta membuat dan bergabung ke tim, serta mengelola struktur tim yang beranggotakan pelari dan supporter. \\
% \hline

% FR-03 & Manajemen Lomba & Sistem mendukung pembuatan dan pengaturan lomba oleh panitia, termasuk jadwal, kategori, aturan, dan konfigurasi event. \\
% \hline

% FR-04 & Pengelolaan Trek & Sistem memungkinkan panitia membuat rute lomba, menetapkan lokasi checkpoint/water station, jam buka, dan cutoff time untuk setiap titik. \\
% \hline

% FR-05 & Pelacakan Peserta Real-time & Sistem menerima dan memproses data lokasi peserta secara kontinu untuk menyediakan posisi terbaru kepada pihak yang berhak. \\
% \hline

% FR-06 & Prediksi ETA & Sistem menghitung ETA peserta berdasarkan pace, progres rute, elevasi, dan dinamika performa pelari. \\
% \hline

% FR-07 & Deteksi Pergantian Pelari & Sistem mendeteksi pergantian pelari pada checkpoint dan mencatat perubahan pelari aktif dalam satu tim. \\
% \hline

% FR-08 & Pencatatan Status Peserta & Sistem mencatat status peserta (DNS, DNF, finished, disqualified, running) serta waktu start, checkpoint, penalti, dan waktu selesai. \\
% \hline

% FR-09 & Penjadwalan Penjemputan & Sistem menampilkan informasi penjemputan, layanan medis, dan drop bag dari panitia kepada peserta secara terorganisir. \\
% \hline

% FR-10 & Permintaan Bantuan & Sistem menyediakan fitur bagi peserta untuk mengirim permintaan pertolongan, berhenti, atau meminta evakuasi kepada panitia. \\
% \hline

% FR-11 & Manajemen Pengguna & Sistem menyediakan autentikasi dan otorisasi, termasuk pengaturan role (peserta, panitia, supporter) serta pembatasan akses sehingga pengguna hanya dapat melihat data sesuai perannya. \\
% \hline

% FR-12 & Notifikasi & Sistem memberikan notifikasi terkait kondisi lomba seperti nearing cutoff time, status darurat, atau perubahan status peserta. \\
% \hline

% FR-13 & Riwayat dan Audit Log & Sistem menyimpan riwayat posisi, kecepatan, status lomba, perubahan oleh panitia, serta log penting lainnya. \\
% \hline

% FR-14 & Dashboard Operasional & Sistem menyediakan dashboard data agregat untuk panitia, termasuk pelari mendekati cutoff dan statistik operasional. \\
% \hline

% FR-15 & Verifikasi Registrasi & Sistem mendukung verifikasi pembayaran atau kelengkapan administrasi sebelum peserta resmi terdaftar. \\
% \hline

% FR-16 & Sinkronisasi Offline & Sistem mendukung buffering data ketika koneksi buruk dan sinkronisasi ulang otomatis ketika koneksi pulih. \\
% \hline

% \end{longtable}
% \end{center}

% functional requirement backend
\begin{center}
\begin{longtable}{|c|p{3.5cm}|p{8.5cm}|}
\caption{Kebutuhan Fungsional Sistem}\label{tab:kebutuhan-fungsional}\\
\hline
\textbf{Kode} & \textbf{Kebutuhan} & \textbf{Deskripsi} \\
\hline
\endfirsthead

\hline
\textbf{Kode} & \textbf{Kebutuhan} & \textbf{Deskripsi} \\
\hline
\endhead

\endfoot

\hline
\endlastfoot

FR-01 & API Pendaftaran Peserta & 
Backend menyediakan API untuk pendaftaran peserta, termasuk validasi data identitas dan penyimpanan informasi dasar peserta ke dalam database. \\
\hline
FR-02 & API Manajemen Tim & 
Backend menyediakan endpoint untuk membuat, mengelola, dan menghubungkan peserta ke dalam tim, termasuk sinkronisasi struktur tim dengan modul lomba dan modul pelacakan. \\
\hline
FR-03 & API Manajemen Lomba & 
Backend menyediakan API bagi panitia untuk membuat, memperbarui, dan mengambil konfigurasi event seperti jadwal, kategori lomba, serta aturan dan parameter lomba. \\
\hline
FR-04 & API Pengelolaan Trek & 
Backend menyediakan API untuk menerima data rute lomba (GPX/TCX), pengaturan checkpoint dan water station, serta pendefinisian jam buka, jam tutup, dan cutoff time untuk setiap titik. \\
\hline
FR-05 & Akuisisi dan Distribusi Data Tracking & 
Backend menerima data lokasi peserta melalui endpoint ingestion dan menyediakan API streaming atau REST untuk menyalurkan data posisi terbaru kepada pihak yang berwenang. \\
\hline
FR-06 & Layanan Backend Perhitungan ETA & 
Backend menjalankan komponen perhitungan ETA berdasarkan pace, progres rute, elevasi, dan dinamika performa pelari, serta menyediakan API untuk pengambilan nilai ETA. \\
\hline
FR-07 & API Deteksi Pergantian Pelari & 
Backend menyediakan mekanisme serta API untuk mencatat dan mendeteksi pergantian pelari dalam satu tim berdasarkan checkpoint atau input panitia. \\
\hline
FR-08 & API Manajemen Status Peserta & 
Backend menyediakan endpoint untuk memperbarui status peserta (DNS, DNF, finished, disqualified, running), mencatat waktu start, checkpoint, penalti, dan waktu selesai. \\
\hline
FR-09 & API Permintaan Bantuan & 
Backend menyediakan endpoint bagi peserta untuk mengirim sinyal bantuan, permintaan berhenti, atau permintaan evakuasi, serta mendistribusikan event tersebut ke modul panitia. \\
\hline
FR-10 & Manajemen Pengguna dan RBAC & 
Backend menyediakan autentikasi, otorisasi, manajemen token, dan pembatasan akses sehingga pengguna hanya dapat melihat data sesuai perannya (panitia, peserta, supporter). \\
\hline
FR-11 & Penyimpanan Riwayat dan Audit Log & 
Backend menyimpan riwayat posisi, kecepatan, perubahan status, aktivitas panitia, serta log penting lainnya yang dapat diakses kembali melalui API. \\
\hline
FR-12 & Sinkronisasi dan Buffering Offline & 
Backend menyediakan mekanisme penerimaan data tertunda akibat koneksi buruk serta melakukan sinkronisasi ulang ketika koneksi pulih. \\
\hline

\end{longtable}
\end{center}


\subsection{Kebutuhan Nonfungsional}
Pada bagian sebelumnya telah dijabarkan kebutuhan fungsional yang menggambarkan perilaku dan layanan utama yang harus disediakan oleh sistem. Namun, pemenuhan fungsi-fungsi tersebut tidak akan berjalan efektif tanpa dukungan kualitas sistem yang memadai. Oleh karena itu, pada bagian ini disusun kebutuhan nonfungsional yang berfokus pada karakteristik teknis, performa, keandalan, keamanan, dan aspek operasional lain yang harus dipenuhi oleh sistem.

Kebutuhan nonfungsional ini memastikan bahwa setiap fungsi yang telah dirumuskan tidak hanya berjalan dengan benar, tetapi juga konsisten, aman, responsif, serta mampu beroperasi dalam skala yang diperlukan oleh kegiatan lomba marathon yang dinamis dan intensif terhadap data. Dengan demikian, bagian ini mendefinisikan standar kualitas yang menjadi fondasi bagi desain arsitektur dan implementasi sistem pada tahap selanjutnya.

\begin{center}
\begin{longtable}{|c|p{3.5cm}|p{8.5cm}|}
\caption{Kebutuhan Nonfungsional Sistem}\label{tab:kebutuhan-nonfungsional}\\
\hline
\textbf{Kode} & \textbf{Kebutuhan} & \textbf{Deskripsi} \\
\hline
\endfirsthead

\hline
\textbf{Kode} & \textbf{Kebutuhan} & \textbf{Deskripsi} \\
\hline
\endhead

\endfoot

\hline
\endlastfoot

NF-01 & Ketersediaan Sistem &
Sistem backend harus memiliki tingkat \textit{availability} minimal 99.9\% selama periode lomba, sehingga layanan inti seperti pelacakan real-time, ETA, dan dashboard panitia tetap dapat diakses tanpa gangguan. \\ \hline

NF-02 & Skalabilitas &
Sistem harus mampu menangani peningkatan jumlah pengguna, klien pelacakan, dan request API secara dinamis dengan mendukung mekanisme autoscaling dan arsitektur terdistribusi di lingkungan cloud. \\ \hline

NF-03 & Performa API &
Endpoint API harus merespons dalam waktu kurang dari 300 ms untuk operasi utama seperti pengambilan posisi pelari, status, dan data ETA, demi menjamin pengalaman real-time. \\ \hline

NF-04 & Konsistensi Data &
Sistem harus memastikan konsistensi data pelari, checkpoint, dan status lomba meskipun terdapat update simultan dari banyak perangkat dan panitia. Mekanisme seperti \textit{event-driven workflow} atau \textit{transactional safety} perlu diterapkan. \\ \hline

NF-05 & Keamanan &
Sistem harus menerapkan autentikasi dan otorisasi berbasis token, enkripsi data dalam transit (HTTPS/TLS), pembatasan akses berdasarkan peran, serta perlindungan terhadap akses tidak sah dan penyalahgunaan API. \\ \hline

NF-06 & Reliabilitas Telemetri &
Sistem harus mampu menerima data lokasi secara kontinu meskipun kondisi jaringan buruk melalui dukungan buffering, retry mechanism, atau strategi \textit{eventual delivery}. Data yang hilang harus diminimalkan. \\ \hline

NF-07 & Integritas ETA &
Model perhitungan ETA harus menghasilkan output yang stabil dan dapat diandalkan, serta tetap akurat meskipun data telemetri tidak lengkap atau mengalami delay. \\ \hline

NF-08 & Fault Tolerance &
Sistem harus tetap beroperasi meskipun terjadi kegagalan pada salah satu komponen (server, pod, database replica, dsb.) dengan menerapkan \textit{redundancy} dan \textit{graceful degradation}. \\ \hline

NF-09 & Monitoring dan Observability &
Sistem menyediakan logging terstruktur, metrik performa, tracing request, dan dashboard monitoring untuk mendeteksi anomali secara cepat. \\ \hline

NF-12 & Kapasitas Penyimpanan &
Sistem harus dapat menyimpan data telemetri dalam jumlah besar (posisi, status, event log) selama periode lomba tanpa mengorbankan performa query. Mekanisme TTL, cold storage, atau kompresi perlu didukung. \\ \hline

\end{longtable}
\end{center}


\section{Analisis Pemilihan Solusi}
Pada tahap sebelumnya telah dijabarkan kebutuhan fungsional dan nonfungsional yang harus dipenuhi oleh sistem pelacakan real-time dan prediksi ETA. Tahap berikutnya adalah menentukan pendekatan teknis yang paling sesuai untuk memenuhi kebutuhan tersebut. Bab ini menyajikan proses pemilihan solusi secara sistematis, dimulai dari identifikasi alternatif yang memungkinkan, analisis komparatif berdasarkan kriteria tertentu, hingga penetapan solusi akhir yang akan digunakan dalam perancangan sistem.

\subsection{Alternatif Solusi}
Bagian ini menguraikan berbagai alternatif solusi yang relevan untuk setiap komponen utama sistem, seperti arsitektur backend, metode akuisisi telemetri, model ETA, penyimpanan data, dan platform deployment. Identifikasi alternatif dilakukan untuk memastikan bahwa seluruh pendekatan yang layak dipertimbangkan secara objektif sebelum dilakukan analisis pemilihan pada tahap berikutnya.

\subsubsection*{Arsitektur Backend}

Beberapa alternatif arsitektur dipertimbangkan untuk mendukung kebutuhan sistem pelacakan real-time dan perhitungan ETA pada ITB Ultra-Marathon. Setiap pendekatan memiliki karakteristik berbeda terkait skalabilitas, kompleksitas, serta kemudahan pengembangan.

\begin{itemize}
    \item Layered Architecture\\
    Pendekatan yang memisahkan berdasarkan lapisan (presentation, service, repository), dengan masing-masing lapisan dapat dideploy secara terpisah (misal front-end, back-end, dan database). Struktur ini mudah dipahami dan cocok untuk tim kecil, serta memungkinkan skalabilitas lebih baik dibanding monolith tunggal, meskipun modul ingest data telemetri mungkin masih menjadi bottleneck jika trafik sangat tinggi.
    
    \item Modular Monolithic\\
    Seluruh fungsi dideploy dalam satu unit, tetapi dipisahkan secara modular berdasarkan domain. 
    Memberikan latensi rendah dan kompleksitas operasional minimal, sekaligus tetap terstruktur.

    \item Microservices\\
    Setiap domain (tracking, event, ETA, tim, notifikasi) berjalan sebagai layanan terpisah. 
    Mendukung penskalaan selektif, namun meningkatkan kebutuhan orkestrasi dan observabilitas.

    \item Event-Driven Architecture\\
    Komunikasi melalui \textit{event bus}. Ideal untuk pemrosesan telemetri, deteksi checkpoint, 
    dan pembaruan ETA secara streaming.
\end{itemize}

\subsubsection*{Protokol Komunikasi Telemetri}

Beberapa opsi protokol dipertimbangkan untuk mengirimkan data lokasi pelari ke backend secara real-time. 
Masing-masing memiliki karakteristik terkait latensi, efisiensi jaringan, dan kompleksitas implementasi.

\begin{itemize}
    \item HTTPS REST Interval-based\\
    Pelari mengirimkan data lokasi setiap X detik melalui permintaan HTTP \texttt{POST}. 
    Pendekatan ini sederhana, mudah diimplementasikan, dan memiliki kompatibilitas tinggi dengan berbagai platform.

    \item WebSocket\\
    Mengirimkan telemetri secara kontinu tanpa overhead \textit{HTTP handshake} berulang. 
    Cocok untuk kebutuhan pelacakan \textit{real-time} dengan interval 1–2 detik dan volume koneksi besar.

    \item MQTT\\
    Protokol IoT yang ringan dan dirancang untuk koneksi jaringan yang tidak stabil. 
    Ideal untuk area rural dengan kualitas internet yang sering fluktuatif.
\end{itemize}

\subsubsection*{Penyimpanan Data Telemetri (Time-Series)}

Beberapa opsi basis data deret-waktu dipertimbangkan untuk menyimpan lokasi pelari, kecepatan, serta data telemetri lainnya. Setiap solusi memiliki kekuatan berbeda dalam hal performa query, skalabilitas, dan kemudahan integrasi.

\begin{itemize}
    \item TimescaleDB (PostgreSQL-based TSDB)\\
    Memudahkan eksekusi query per pelari, jarak, maupun checkpoint. 
    Cocok digunakan untuk beban menengah dengan kebutuhan analisis yang fleksibel.

    \item InfluxDB\\
    Basis data yang dioptimalkan khusus untuk data \textit{time-series}. 
    Memberikan performa query sangat cepat untuk kebutuhan analisis historis.
\end{itemize}

\subsubsection*{Platform Deployment}

Berbagai opsi platform deployment dipertimbangkan untuk menyeimbangkan fleksibilitas, skalabilitas, dan kompleksitas operasional sistem.

\begin{itemize}
    \item Cloud VM (IaaS)\\
    Menggunakan layanan seperti GCP Compute Engine atau AWS EC2. 
    Memberikan kontrol penuh terhadap lingkungan dan konfigurasi, serta menawarkan fleksibilitas tertinggi.

    \item Container Orchestration (Kubernetes)\\
    Cocok untuk arsitektur dengan banyak \textit{service}. 
    Mendukung \textit{auto-scaling} pada modul ingest telemetri dan memudahkan pengelolaan deployment berskala besar.

    \item Serverless\\
    Menggunakan layanan seperti Cloud Run atau AWS Lambda. 
    Ideal untuk modul yang berjalan tidak kontinu, seperti layanan ETA API, notifikasi, atau API pendaftaran.
\end{itemize}

\subsubsection*{Monitoring (Metrics)}

Sistem memerlukan pemantauan metrik untuk memastikan proses ingest telemetri, perhitungan ETA, dan API tetap beroperasi secara optimal. Beberapa opsi monitoring yang dipertimbangkan adalah:

\begin{itemize}
    \item Prometheus + Grafana\\
    Solusi open-source standar industri. 
    Mendukung pemantauan metrik seperti \textit{ingestion rate}, \textit{ETA latency}, dan \textit{API throughput}.

    \item Datadog Metrics\\
    Layanan \textit{managed} yang menawarkan integrasi cepat dan operasional sederhana. 
    Cocok untuk acara jangka pendek yang membutuhkan setup minim.

    \item Cloud-Native Monitoring (AWS CloudWatch / GCP Cloud Monitoring)\\
    Integrasi langsung dengan layanan cloud tanpa perlu instalasi tambahan. 
    Mendukung pengumpulan metrik, log, dan \textit{alerts} secara otomatis untuk VM, Kubernetes, maupun serverless.
\end{itemize}

\subsubsection*{Logging (Event \& System Logs)}

Sistem membutuhkan mekanisme pencatatan log yang andal untuk debugging, audit, dan analisis event selama lomba berlangsung. Beberapa opsi yang dipertimbangkan antara lain:

\begin{itemize}
    \item ELK Stack (Elasticsearch + Logstash + Kibana)\\
    Solusi populer yang sangat powerful untuk pencarian dan analisis log skala besar. 
    Cocok untuk kebutuhan \textit{event tracking} yang kompleks.

    \item Loki + Grafana\\
    Pendekatan yang lebih ringan dengan fokus pada \textit{log streaming}. 
    Sederhana, efisien, dan lebih hemat sumber daya dibanding ELK.

    \item Cloud-Native Logging (AWS CloudWatch Logs / GCP Cloud Logging)\\
    Integrasi bawaan dari platform cloud. 
    Mendukung pengumpulan log otomatis dari VM, Kubernetes, maupun layanan serverless, 
    dengan konfigurasi minimal dan \textit{managed retention}.
\end{itemize}

\subsubsection*{Tracing (Distributed Tracing)}

Tracing diperlukan untuk menganalisis alur request antar layanan, terutama jika sistem menggunakan pendekatan microservice atau modul-modul terpisah. Beberapa opsi tracing yang dipertimbangkan:

\begin{itemize}
    \item Jaeger\\
    Solusi open-source populer untuk \textit{distributed tracing}. 
    Jaeger dapat diintegrasikan dengan OpenTelemetry sebagai layer instrumentasi, maupun dengan ekosistem Prometheus, Grafana, dan Loki (melalui Grafana Tempo atau dashboard Jaeger). 
    Cocok untuk sistem dengan banyak layanan yang saling berkomunikasi.

    \item Datadog APM\\
    Layanan \textit{managed} dengan instalasi sederhana dan integrasi otomatis. 
    Menyediakan tampilan \textit{end-to-end tracing} dengan konfigurasi minimal, ideal untuk event jangka pendek.

    \item Cloud-Native Tracing (AWS X-Ray / GCP Cloud Trace)\\
    Layanan tracing terkelola penuh dari penyedia cloud. 
    Terintegrasi langsung dengan layanan cloud-native seperti serverless, load balancer, dan service mesh, 
    sehingga mengurangi overhead operasional dalam pengelolaan infrastruktur tracing.
\end{itemize}

\subsubsection*{Model Perhitungan ETA Marathon}

Beberapa pendekatan umum yang banyak digunakan pada aplikasi lari dan perangkat GPS dalam menghitung estimasi waktu tiba (ETA) pelari marathon.

\begin{itemize}
    \item Ekstrapolasi Pace Rata-Rata\\
    Menggunakan pace rata-rata sejauh ini untuk memproyeksikan waktu tempuh sisa jarak. 
    Sederhana dan cepat, namun tidak mempertimbangkan kelelahan atau variasi medan.

    \item Model Pace Berbasis Segmen\\
    Mengambil pace terbaru (mis.\ beberapa kilometer terakhir) dan memprediksi sisa jarak per segmen. 
    Dapat memasukkan faktor pelambatan bertahap atau kondisi rute sehingga hasil lebih realistis.

    \item Model Berbasis Data Historis\\
    Menggabungkan data performa sebelumnya, kondisi lingkungan, serta profil rute dalam model regresi atau pembelajaran mesin. 
    Lebih personal dan akurat, tetapi memerlukan data dan proses kalibrasi tambahan.
\end{itemize}

\subsubsection*{Integrasi Elevasi dalam Perhitungan ETA}

Elevasi merupakan faktor penting dalam memproyeksikan performa pelari karena tanjakan dan turunan 
dapat mengubah kebutuhan usaha secara signifikan. Beberapa pendekatan umum untuk memasukkan faktor ini adalah:

\begin{itemize}
    \item Grade-Adjusted Pace (GAP)\\
    Mengoreksi pace aktual berdasarkan kemiringan lintasan 
    sehingga waktu tempuh pada segmen menanjak atau menurun dapat disetarakan dengan kondisi datar. 
    Hasil pace yang telah dikoreksi digunakan dalam model ETA apa pun.

    \item Elevation-Corrected Segment ETA\\
    Menghitung ETA per segmen dengan menambahkan penalti atau bonus waktu 
    bergantung pada elevasi segmen tersebut. Pendekatan ini umum pada aplikasi tracking berbasis rute.
\end{itemize}


\subsection{Analisis Penentuan Solusi}
Analisis penentuan solusi pada penelitian ini menggunakan metode Multi-Criteria Decision Analysis (MCDA) dengan pendekatan Weighted Scoring. Metode ini dipilih karena mampu mengevaluasi alternatif solusi teknis berdasarkan bobot kriteria yang relevan seperti performa, skalabilitas, reliabilitas, kompleksitas implementasi, dan biaya.

\subsubsection*{Arsitektur Backend}
\begin{itemize}
    \item \textbf{Skalabilitas} (35\%): Kemampuan arsitektur untuk menangani peningkatan jumlah data dan pengguna secara efisien.
    \item \textbf{Kompleksitas Pengembangan dan Operasional} (25\%): Tingkat kesulitan implementasi dan pemeliharaan sistem.
    \item \textbf{Latensi Pemrosesan Telemetri} (25\%): Kecepatan arsitektur dalam memproses data telemetri secara real-time.
    \item \textbf{Kesesuaian dengan Real-Time Event/Streaming} (15\%): Kemampuan arsitektur untuk mendukung aliran data berbasis event secara kontinu.
\end{itemize}

Penilaian tiap alternatif dilakukan dengan skala 1--10, di mana nilai lebih tinggi menunjukkan performa yang lebih baik. Hasil penilaian disajikan pada tabel berikut:

\begin{table}[H]
\centering
\caption{Penilaian Tiap Alternatif Arsitektur Backend}
\begin{tabular}{lcccc}
\hline
\textbf{Arsitektur} & \textbf{Skalabilitas} & \textbf{Kompleksitas} & \textbf{Latensi} & \textbf{Real-Time Events} \\
\hline
Layered Architecture & 6 & 8 & 7 & 5 \\
Modular Monolithic & 4 & 8 & 8 & 7 \\
Microservices & 9 & 5 & 8 & 9 \\
Event-Driven Architecture & 9 & 6 & 9 & 10 \\
\hline
\end{tabular}
\end{table}

Berdasarkan bobot kriteria, skor tertimbang tiap alternatif dihitung dengan rumus:
\[
\text{Skor Total} = \sum (\text{nilai kriteria} \times \text{bobot})
\]

\begin{table}[H]
\centering
\caption{Hasil Perhitungan Weighted Score Arsitektur Backend}
\begin{tabular}{lcc}
\hline
\textbf{Arsitektur} & \textbf{Skor Total} & \textbf{Ranking} \\
\hline
Event-Driven Architecture & 8.40 & 1 \\
Microservices & 7.75 & 2 \\
Layered Architecture & 6.60 & 3 \\
Modular Monolithic & 6.45 & 4 \\
\hline
\end{tabular}
\end{table}

Berdasarkan hasil MCDA weighted scoring, \textbf{Event-Driven Architecture} merupakan arsitektur backend yang paling sesuai untuk sistem ITB Ultra-Marathon. Pendekatan ini unggul dalam menangani data telemetri secara real-time, mendukung aliran event streaming, serta memiliki kemampuan skalabilitas tinggi. Meskipun kompleksitas pengembangan sedikit lebih tinggi dibandingkan Layered Architecture dan Modular Monolithic, keuntungan dalam performa real-time dan fleksibilitas penskalaan membuat Event-Driven Architecture menjadi pilihan optimal. Microservices menjadi alternatif kedua yang layak dipertimbangkan jika modularitas layanan terpisah menjadi prioritas. 

\subsubsection*{Protokol Komunikasi Telemetri}

Untuk menentukan protokol komunikasi telemetri yang paling sesuai dalam pengiriman data lokasi pelari ke backend secara real-time, digunakan pendekatan Multi-Criteria Decision Analysis (MCDA) dengan metode weighted scoring. Kriteria yang digunakan beserta bobotnya adalah sebagai berikut:

\begin{itemize}
    \item \textbf{Latensi Pengiriman Data} (35\%): Kecepatan protokol dalam menyampaikan data lokasi secara real-time.
    \item \textbf{Efisiensi Jaringan} (30\%): Penggunaan bandwidth dan overhead protokol dalam kondisi jaringan terbatas.
    \item \textbf{Kompleksitas Implementasi} (20\%): Tingkat kesulitan pengembangan dan integrasi protokol.
    \item \textbf{Kestabilan Koneksi pada Kondisi Fluktuatif} (15\%): Kemampuan protokol menjaga koneksi stabil di area dengan kualitas jaringan tidak konsisten.
\end{itemize}

Penilaian tiap protokol dilakukan dengan skala 1--10, di mana nilai lebih tinggi menunjukkan performa yang lebih baik. Hasil penilaian disajikan pada tabel berikut:

\begin{table}[H]
\centering
\caption{Penilaian Tiap Alternatif Protokol Komunikasi Telemetri}
\begin{tabular}{lcccc}
\hline
\textbf{Protokol} & \textbf{Latensi} & \textbf{Efisiensi Jaringan} & \textbf{Kompleksitas} & \textbf{Kestabilan Koneksi} \\
\hline
HTTPS REST Interval-based & 6 & 7 & 9 & 6 \\
WebSocket & 9 & 8 & 7 & 7 \\
MQTT & 8 & 9 & 8 & 9 \\
\hline
\end{tabular}
\end{table}

Berdasarkan bobot kriteria, skor tertimbang tiap alternatif dihitung dengan rumus:
\[
\text{Skor Total} = \sum (\text{nilai kriteria} \times \text{bobot})
\]

\begin{table}[H]
\centering
\caption{Hasil Perhitungan Weighted Score Protokol Komunikasi Telemetri}
\begin{tabular}{lcc}
\hline
\textbf{Protokol} & \textbf{Skor Total} & \textbf{Ranking} \\
\hline
MQTT & 8.50 & 1 \\
WebSocket & 8.10 & 2 \\
HTTPS REST Interval-based & 7.05 & 3 \\
\hline
\end{tabular}
\end{table}

Berdasarkan hasil MCDA weighted scoring, \textbf{MQTT} menjadi protokol komunikasi telemetri yang paling sesuai untuk sistem pelacakan ITB Ultra-Marathon. Protokol ini unggul dalam efisiensi jaringan dan kestabilan koneksi di area dengan kualitas internet yang berfluktuasi, serta memiliki latensi yang rendah dan kompleksitas implementasi yang moderat. WebSocket menjadi alternatif kedua yang baik untuk kebutuhan real-time dengan volume koneksi besar, sementara HTTPS REST Interval-based tetap layak dipertimbangkan untuk implementasi sederhana dan kompatibilitas platform yang tinggi.

\subsubsection*{Penyimpanan Data Telemetri (Time-Series)}

Untuk menentukan solusi basis data deret-waktu yang paling sesuai dalam menyimpan lokasi pelari, kecepatan, dan data telemetri lainnya, digunakan pendekatan Multi-Criteria Decision Analysis (MCDA) dengan metode weighted scoring. Kriteria yang digunakan beserta bobotnya adalah sebagai berikut:

\begin{itemize}
    \item \textbf{Performa Query} (40\%): Kecepatan eksekusi query, baik untuk analisis historis maupun per pelari/checkpoint.
    \item \textbf{Skalabilitas} (30\%): Kemampuan basis data menangani peningkatan jumlah data dan pengguna secara efisien.
    \item \textbf{Kemudahan Integrasi dan Pemeliharaan} (30\%): Tingkat kesulitan integrasi dengan sistem backend dan kompleksitas operasional.
\end{itemize}

Penilaian tiap alternatif dilakukan dengan skala 1--10, di mana nilai lebih tinggi menunjukkan performa yang lebih baik. Hasil penilaian disajikan pada tabel berikut:

\begin{table}[H]
\centering
\caption{Penilaian Tiap Alternatif Basis Data Time-Series}
\begin{tabular}{lccc}
\hline
\textbf{Basis Data} & \textbf{Performa Query} & \textbf{Skalabilitas} & \textbf{Integrasi \& Pemeliharaan} \\
\hline
TimescaleDB & 8 & 7 & 8 \\
InfluxDB & 9 & 8 & 7 \\
\hline
\end{tabular}
\end{table}

Berdasarkan bobot kriteria, skor tertimbang tiap alternatif dihitung dengan rumus:
\[
\text{Skor Total} = \sum (\text{nilai kriteria} \times \text{bobot})
\]

\begin{table}[H]
\centering
\caption{Hasil Perhitungan Weighted Score Basis Data Time-Series}
\begin{tabular}{lcc}
\hline
\textbf{Basis Data} & \textbf{Skor Total} & \textbf{Ranking} \\
\hline
InfluxDB & 8.20 & 1 \\
TimescaleDB & 7.75 & 2 \\
\hline
\end{tabular}
\end{table}

Berdasarkan hasil MCDA weighted scoring, \textbf{InfluxDB} merupakan pilihan basis data deret-waktu yang paling sesuai untuk penyimpanan telemetri ITB Ultra-Marathon. InfluxDB unggul dalam performa query untuk analisis historis dan mampu diskalakan dengan baik seiring pertumbuhan jumlah data. TimescaleDB tetap menjadi alternatif yang layak dipertimbangkan jika kebutuhan analisis fleksibel per pelari atau checkpoint lebih diutamakan, serta jika integrasi dengan ekosistem PostgreSQL menjadi prioritas.

\subsubsection*{Platform Deployment}

Untuk menentukan platform deployment yang paling sesuai bagi sistem ITB Ultra-Marathon, digunakan pendekatan Multi-Criteria Decision Analysis (MCDA) dengan metode weighted scoring. Kriteria yang digunakan beserta bobotnya adalah sebagai berikut:

\begin{itemize}
    \item \textbf{Fleksibilitas Konfigurasi} (30\%): Kemampuan platform untuk mendukung konfigurasi khusus dan pengaturan lingkungan sesuai kebutuhan.
    \item \textbf{Skalabilitas} (35\%): Kemampuan platform menangani peningkatan beban, baik dari sisi jumlah pengguna maupun volume data telemetri.
    \item \textbf{Kompleksitas Operasional} (20\%): Tingkat kesulitan dalam pengelolaan, pemeliharaan, dan monitoring sistem.
    \item \textbf{Efisiensi Biaya} (15\%): Perbandingan biaya operasional relatif terhadap performa dan kapasitas yang diperoleh.
\end{itemize}

Penilaian tiap alternatif dilakukan dengan skala 1--10, di mana nilai lebih tinggi menunjukkan performa yang lebih baik. Hasil penilaian disajikan pada tabel berikut:

\begin{table}[H]
\centering
\caption{Penilaian Tiap Alternatif Platform Deployment}
\begin{tabular}{lcccc}
\hline
\textbf{Platform} & \textbf{Fleksibilitas} & \textbf{Skalabilitas} & \textbf{Kompleksitas Operasional} & \textbf{Efisiensi Biaya} \\
\hline
Cloud VM (IaaS) & 9 & 7 & 6 & 6 \\
Container Orchestration (Kubernetes) & 8 & 9 & 7 & 7 \\
Serverless & 6 & 8 & 9 & 8 \\
\hline
\end{tabular}
\end{table}

Berdasarkan bobot kriteria, skor tertimbang tiap alternatif dihitung dengan rumus:
\[
\text{Skor Total} = \sum (\text{nilai kriteria} \times \text{bobot})
\]

\begin{table}[H]
\centering
\caption{Hasil Perhitungan Weighted Score Platform Deployment}
\begin{tabular}{lcc}
\hline
\textbf{Platform} & \textbf{Skor Total} & \textbf{Ranking} \\
\hline
Container Orchestration (Kubernetes) & 8.50 & 1 \\
Serverless & 7.95 & 2 \\
Cloud VM (IaaS) & 7.25 & 3 \\
\hline
\end{tabular}
\end{table}

Berdasarkan hasil MCDA weighted scoring, \textbf{Container Orchestration (Kubernetes)} menjadi pilihan platform deployment yang paling sesuai untuk sistem ITB Ultra-Marathon. Platform ini unggul dalam hal skalabilitas dan fleksibilitas konfigurasi, serta memudahkan pengelolaan layanan berskala besar. Serverless merupakan alternatif yang baik untuk modul yang berjalan tidak kontinu, dengan efisiensi biaya tinggi dan kompleksitas operasional rendah. Cloud VM (IaaS) tetap layak dipertimbangkan jika kontrol penuh terhadap lingkungan dan konfigurasi menjadi prioritas, meskipun skalabilitas dan efisiensi biayanya lebih terbatas dibandingkan opsi lainnya.

\subsubsection*{Monitoring Metrics}

Untuk memastikan proses ingest telemetri, perhitungan ETA, dan API beroperasi secara optimal, sistem memerlukan pemantauan metrik yang efektif. Pendekatan Multi-Criteria Decision Analysis (MCDA) dengan metode weighted scoring digunakan untuk menilai beberapa opsi monitoring. Kriteria yang digunakan beserta bobotnya adalah sebagai berikut:

\begin{itemize}
    \item \textbf{Kemudahan Integrasi} (30\%): Tingkat kesederhanaan integrasi sistem monitoring dengan platform yang digunakan.
    \item \textbf{Kelengkapan Fitur Metrik} (35\%): Kemampuan sistem untuk memantau berbagai metrik, termasuk ingestion rate, ETA latency, dan API throughput.
    \item \textbf{Kompleksitas Operasional} (20\%): Tingkat kesulitan dalam pengelolaan, konfigurasi, dan pemeliharaan monitoring.
    \item \textbf{Biaya} (15\%): Efisiensi biaya dalam penggunaan sistem monitoring, termasuk lisensi dan operasional.
\end{itemize}

Penilaian tiap alternatif dilakukan dengan skala 1--10, di mana nilai lebih tinggi menunjukkan performa yang lebih baik. Hasil penilaian disajikan pada tabel berikut:

\begin{table}[H]
\centering
\caption{Penilaian Tiap Alternatif Sistem Monitoring Metrics}
\begin{tabular}{lcccc}
\hline
\textbf{Sistem Monitoring} & \textbf{Kemudahan Integrasi} & \textbf{Kelengkapan Fitur Metrik} & \textbf{Kompleksitas Operasional} & \textbf{Biaya} \\
\hline
Prometheus + Grafana & 7 & 9 & 7 & 8 \\
Datadog Metrics & 9 & 8 & 9 & 6 \\
Cloud-Native Monitoring & 8 & 8 & 8 & 7 \\
\hline
\end{tabular}
\end{table}

Berdasarkan bobot kriteria, skor tertimbang tiap alternatif dihitung dengan rumus:
\[
\text{Skor Total} = \sum (\text{nilai kriteria} \times \text{bobot})
\]

\begin{table}[H]
\centering
\caption{Hasil Perhitungan Weighted Score Sistem Monitoring Metrics}
\begin{tabular}{lcc}
\hline
\textbf{Sistem Monitoring} & \textbf{Skor Total} & \textbf{Ranking} \\
\hline
Prometheus + Grafana & 8.15 & 1 \\
Cloud-Native Monitoring & 7.85 & 2 \\
Datadog Metrics & 7.80 & 3 \\
\hline
\end{tabular}
\end{table}

Berdasarkan hasil MCDA weighted scoring, \textbf{Prometheus + Grafana} merupakan pilihan sistem monitoring yang paling sesuai untuk ITB Ultra-Marathon. Solusi ini unggul dalam hal kelengkapan fitur metrik dan memiliki kompleksitas operasional yang moderat, sehingga memungkinkan pemantauan ingest telemetri, perhitungan ETA, dan API secara efektif. Cloud-Native Monitoring menjadi alternatif kedua dengan kemudahan integrasi dan biaya yang seimbang, sedangkan Datadog Metrics cocok untuk acara jangka pendek yang membutuhkan setup cepat meskipun biaya relatif lebih tinggi.

\subsubsection*{Logging (Event \& System Logs)}

Untuk mendukung debugging, audit, dan analisis event selama lomba ITB Ultra-Marathon, sistem memerlukan mekanisme pencatatan log yang andal. Pendekatan Multi-Criteria Decision Analysis (MCDA) dengan metode weighted scoring digunakan untuk menilai beberapa opsi sistem logging. Kriteria yang digunakan beserta bobotnya adalah sebagai berikut:

\begin{itemize}
    \item \textbf{Kemampuan Analisis dan Pencarian Log} (35\%): Kapabilitas sistem dalam melakukan pencarian, agregasi, dan analisis log secara efektif.
    \item \textbf{Kompleksitas Operasional} (25\%): Tingkat kesulitan dalam instalasi, konfigurasi, pemeliharaan, dan monitoring sistem logging.
    \item \textbf{Integrasi dengan Infrastruktur} (20\%): Kemudahan integrasi dengan VM, Kubernetes, serverless, dan layanan lain yang digunakan.
    \item \textbf{Efisiensi Sumber Daya / Biaya} (20\%): Penggunaan sumber daya dan biaya operasional relatif terhadap performa sistem.
\end{itemize}

Penilaian tiap alternatif dilakukan dengan skala 1--10, di mana nilai lebih tinggi menunjukkan performa yang lebih baik. Hasil penilaian disajikan pada tabel berikut:

\begin{table}[H]
\centering
\caption{Penilaian Tiap Alternatif Sistem Logging}
\begin{tabular}{lcccc}
\hline
\textbf{Sistem Logging} & \textbf{Analisis \& Pencarian} & \textbf{Kompleksitas Operasional} & \textbf{Integrasi Infrastruktur} & \textbf{Efisiensi Sumber Daya / Biaya} \\
\hline
ELK Stack & 9 & 6 & 7 & 6 \\
Loki + Grafana & 7 & 8 & 8 & 8 \\
Cloud-Native Logging & 8 & 8 & 9 & 7 \\
\hline
\end{tabular}
\end{table}

Berdasarkan bobot kriteria, skor tertimbang tiap alternatif dihitung dengan rumus:
\[
\text{Skor Total} = \sum (\text{nilai kriteria} \times \text{bobot})
\]

\begin{table}[H]
\centering
\caption{Hasil Perhitungan Weighted Score Sistem Logging}
\begin{tabular}{lcc}
\hline
\textbf{Sistem Logging} & \textbf{Skor Total} & \textbf{Ranking} \\
\hline
Cloud-Native Logging & 8.15 & 1 \\
Loki + Grafana & 7.75 & 2 \\
ELK Stack & 7.65 & 3 \\
\hline
\end{tabular}
\end{table}

Berdasarkan hasil MCDA weighted scoring, \textbf{Cloud-Native Logging} menjadi pilihan sistem logging yang paling sesuai untuk ITB Ultra-Marathon. Solusi ini menawarkan integrasi yang mudah dengan infrastruktur yang digunakan, kompleksitas operasional moderat, serta kemampuan analisis log yang memadai. Loki + Grafana menjadi alternatif kedua yang efisien dan hemat sumber daya, sedangkan ELK Stack cocok untuk kebutuhan analisis log yang kompleks, tetapi memerlukan sumber daya dan konfigurasi lebih tinggi.


\subsubsection*{Tracing (Distributed Tracing)}

Untuk menganalisis alur request antar layanan, terutama pada arsitektur microservices, sistem memerlukan mekanisme distributed tracing yang efektif. Pendekatan Multi-Criteria Decision Analysis (MCDA) dengan metode weighted scoring digunakan untuk menilai beberapa opsi. Kriteria dan bobotnya adalah sebagai berikut:

\begin{itemize}
    \item \textbf{Kemampuan End-to-End Tracing} (35\%): Kapabilitas sistem untuk melacak alur request antar layanan secara lengkap.
    \item \textbf{Kemudahan Integrasi} (25\%): Tingkat kesederhanaan integrasi dengan layanan dan sistem monitoring/metrik yang digunakan.
    \item \textbf{Kompleksitas Operasional} (20\%): Tingkat kesulitan instalasi, konfigurasi, dan pemeliharaan sistem tracing.
    \item \textbf{Biaya / Efisiensi} (20\%): Efisiensi biaya relatif terhadap fitur dan kapasitas tracing yang diberikan.
\end{itemize}

Penilaian tiap alternatif dilakukan dengan skala 1--10, di mana nilai lebih tinggi menunjukkan performa yang lebih baik. Hasil penilaian disajikan pada tabel berikut:

\begin{table}[H]
\centering
\caption{Penilaian Tiap Alternatif Sistem Tracing}
\begin{tabular}{lcccc}
\hline
\textbf{Sistem Tracing} & \textbf{End-to-End Tracing} & \textbf{Kemudahan Integrasi} & \textbf{Kompleksitas Operasional} & \textbf{Biaya / Efisiensi} \\
\hline
Jaeger & 9 & 7 & 7 & 7 \\
Datadog APM & 8 & 9 & 9 & 6 \\
Cloud-Native Tracing & 8 & 8 & 8 & 7 \\
\hline
\end{tabular}
\end{table}

Berdasarkan bobot kriteria, skor tertimbang tiap alternatif dihitung dengan rumus:
\[
\text{Skor Total} = \sum (\text{nilai kriteria} \times \text{bobot})
\]

\begin{table}[H]
\centering
\caption{Hasil Perhitungan Weighted Score Sistem Tracing}
\begin{tabular}{lcc}
\hline
\textbf{Sistem Tracing} & \textbf{Skor Total} & \textbf{Ranking} \\
\hline
Jaeger & 8.10 & 1 \\
Cloud-Native Tracing & 7.90 & 2 \\
Datadog APM & 7.80 & 3 \\
\hline
\end{tabular}
\end{table}

Berdasarkan hasil MCDA weighted scoring, \textbf{Jaeger} menjadi pilihan utama untuk sistem distributed tracing ITB Ultra-Marathon. Solusi ini unggul dalam kemampuan end-to-end tracing antar layanan dan integrasi dengan ekosistem observabilitas seperti Prometheus, Grafana, dan Loki. Cloud-Native Tracing menjadi alternatif kedua yang baik dengan kemudahan integrasi dan kompleksitas moderat, sedangkan Datadog APM cocok untuk setup cepat dan event jangka pendek meskipun biaya relatif lebih tinggi.


\subsubsection*{Model Perhitungan ETA Marathon}

Untuk menentukan model perhitungan estimasi waktu tiba (ETA) pelari marathon yang paling sesuai, digunakan pendekatan Multi-Criteria Decision Analysis (MCDA) dengan metode weighted scoring. Kriteria yang digunakan beserta bobotnya adalah sebagai berikut:

\begin{itemize}
    \item \textbf{Akurasi Prediksi} (40\%): Kemampuan model memprediksi ETA secara realistis, termasuk mempertimbangkan variasi pace dan kondisi rute.
    \item \textbf{Kompleksitas Implementasi} (25\%): Tingkat kesulitan pengembangan, integrasi, dan kalibrasi model.
    \item \textbf{Kebutuhan Data} (20\%): Ketersediaan data yang dibutuhkan untuk menjalankan model secara efektif.
    \item \textbf{Kecepatan Perhitungan} (15\%): Waktu yang dibutuhkan model untuk menghitung ETA secara real-time.
\end{itemize}

Penilaian tiap alternatif dilakukan dengan skala 1--10, di mana nilai lebih tinggi menunjukkan performa yang lebih baik. Hasil penilaian disajikan pada tabel berikut:

\begin{table}[H]
\centering
\caption{Penilaian Tiap Alternatif Model Perhitungan ETA}
\begin{tabular}{lcccc}
\hline
\textbf{Model ETA} & \textbf{Akurasi Prediksi} & \textbf{Kompleksitas Implementasi} & \textbf{Kebutuhan Data} & \textbf{Kecepatan Perhitungan} \\
\hline
Ekstrapolasi Pace Rata-Rata & 6 & 9 & 9 & 9 \\
Model Pace Berbasis Segmen & 8 & 7 & 7 & 8 \\
Model Berbasis Data Historis & 9 & 5 & 6 & 6 \\
\hline
\end{tabular}
\end{table}

Berdasarkan bobot kriteria, skor tertimbang tiap alternatif dihitung dengan rumus:
\[
\text{Skor Total} = \sum (\text{nilai kriteria} \times \text{bobot})
\]

\begin{table}[H]
\centering
\caption{Hasil Perhitungan Weighted Score Model ETA}
\begin{tabular}{lcc}
\hline
\textbf{Model ETA} & \textbf{Skor Total} & \textbf{Ranking} \\
\hline
Model Pace Berbasis Segmen & 7.70 & 1 \\
Model Berbasis Data Historis & 7.05 & 2 \\
Ekstrapolasi Pace Rata-Rata & 7.05 & 3 \\
\hline
\end{tabular}
\end{table}

Berdasarkan hasil MCDA weighted scoring, \textbf{Model Pace Berbasis Segmen} menjadi pilihan utama untuk perhitungan ETA pelari marathon. Model ini menawarkan akurasi yang baik dengan kompleksitas implementasi moderat, serta dapat memperhitungkan variasi pace dan kondisi rute per segmen. Model Berbasis Data Historis memberikan prediksi yang sangat akurat namun memerlukan data tambahan dan proses kalibrasi, sedangkan Ekstrapolasi Pace Rata-Rata sederhana dan cepat, tetapi kurang mempertimbangkan kelelahan atau variasi medan.


\subsubsection*{Integrasi Elevasi dalam Perhitungan ETA}

Untuk memperhitungkan faktor elevasi yang mempengaruhi performa pelari, digunakan pendekatan Multi-Criteria Decision Analysis (MCDA) dengan metode weighted scoring. Kriteria yang digunakan beserta bobotnya adalah sebagai berikut:

\begin{itemize}
    \item \textbf{Akurasi Perhitungan ETA} (40\%): Kemampuan pendekatan untuk menghasilkan estimasi waktu tiba yang realistis dengan mempertimbangkan tanjakan dan turunan.
    \item \textbf{Kompleksitas Implementasi} (30\%): Tingkat kesulitan integrasi pendekatan ke dalam model ETA yang digunakan.
    \item \textbf{Kebutuhan Data Elevasi} (20\%): Ketersediaan data elevasi yang diperlukan dan kemudahan penggunaannya.
    \item \textbf{Kecepatan Perhitungan} (10\%): Waktu yang dibutuhkan pendekatan untuk menghitung ETA secara real-time.
\end{itemize}

Penilaian tiap alternatif dilakukan dengan skala 1--10, di mana nilai lebih tinggi menunjukkan performa yang lebih baik. Hasil penilaian disajikan pada tabel berikut:

\begin{table}[H]
\centering
\caption{Penilaian Tiap Alternatif Pendekatan Integrasi Elevasi}
\begin{tabular}{lcccc}
\hline
\textbf{Pendekatan} & \textbf{Akurasi ETA} & \textbf{Kompleksitas Implementasi} & \textbf{Kebutuhan Data} & \textbf{Kecepatan Perhitungan} \\
\hline
Grade-Adjusted Pace (GAP) & 7 & 8 & 8 & 9 \\
Elevation-Corrected Segment ETA & 9 & 6 & 7 & 7 \\
\hline
\end{tabular}
\end{table}

Berdasarkan bobot kriteria, skor tertimbang tiap alternatif dihitung dengan rumus:
\[
\text{Skor Total} = \sum (\text{nilai kriteria} \times \text{bobot})
\]

\begin{table}[H]
\centering
\caption{Hasil Perhitungan Weighted Score Pendekatan Integrasi Elevasi}
\begin{tabular}{lcc}
\hline
\textbf{Pendekatan} & \textbf{Skor Total} & \textbf{Ranking} \\
\hline
Elevation-Corrected Segment ETA & 7.95 & 1 \\
Grade-Adjusted Pace (GAP) & 7.70 & 2 \\
\hline
\end{tabular}
\end{table}

Berdasarkan hasil MCDA weighted scoring, \textbf{Elevation-Corrected Segment ETA} menjadi pendekatan yang lebih unggul untuk integrasi faktor elevasi dalam perhitungan ETA pelari marathon. Pendekatan ini memberikan akurasi prediksi yang lebih tinggi dengan mempertimbangkan penalti atau bonus waktu per segmen. Grade-Adjusted Pace (GAP) tetap layak dipertimbangkan karena kompleksitas implementasinya lebih rendah dan perhitungannya lebih cepat, namun akurasi prediksi sedikit lebih rendah dibandingkan pendekatan per segmen.
