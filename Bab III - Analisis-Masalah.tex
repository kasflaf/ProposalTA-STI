% ============================================================================================
% BAB III ANALISIS MASALAH
% Pembagian subbab tidak rigid dan dapat bervariasi. Bab ini minimal berisi analisis kebutuhan
% fungsional dan nonfungsional, analisis berbagai alternatif solusi yang dapat ditawarkan, dan
% metode pemilihan solusi yang diusulkan.
% ============================================================================================
\chapter{ANALISIS MASALAH}
\label{chap:analisis-masalah}
\section{Analisis Kondisi Saat Ini}

Seiring dengan meningkatnya kebutuhan akan sistem pelacakan pelari yang akurat dan dapat diakses secara \textit{real-time} dalam penyelenggaraan ITB Ultra-Marathon, penting untuk memahami terlebih dahulu kondisi teknologi dan pendekatan yang saat ini digunakan di industri maupun penelitian. Analisis ini diperlukan untuk mengidentifikasi bagaimana sistem pencatatan waktu konvensional bekerja, sejauh mana teknologi pelacakan kontinu telah berkembang, serta apa saja keterbatasan pada metode prediksi ETA yang tersedia. Pemahaman menyeluruh terhadap kondisi eksisting menjadi dasar dalam merumuskan rancangan \textit{backend} yang mampu menangani skala besar, mendukung prediksi yang lebih kontekstual, dan memenuhi kebutuhan operasional maraton secara spesifik.

\subsection{Model Sistem Berbasis RFID}

Standar industri dalam manajemen lomba lari massal saat ini adalah sistem \textit{timing} berbasis \textit{Radio Frequency Identification} (RFID), khususnya RFID pasif UHF. Model konseptual sistem ini mengandalkan interaksi antara \textit{tag} yang dibawa pelari dengan infrastruktur pembaca yang dipasang pada titik-titik tertentu \cite{RaceID2022}.

Komponen utama dalam sistem ini meliputi:
\begin{enumerate}
    \item Transponder Pasif (Tag): \textit{Chip} tanpa baterai yang ditempelkan pada nomor dada pelari dan hanya aktif ketika menerima energi dari pembaca.
    \item Infrastruktur \textit{Reader/Mats}: Antena atau karpet pembaca yang ditempatkan pada titik diskrit seperti garis \textit{start}, \textit{split} setiap beberapa kilometer, dan garis \textit{finish}.
    \item Dekoder dan Perangkat Lunak: Perangkat yang mengonversi sinyal menjadi ID pelari dan stempel waktu untuk diproses lebih lanjut.
\end{enumerate}

Sistem RFID memiliki keterbatasan mendasar berupa sifat data yang diskrit. Posisi pelari hanya diketahui ketika melewati titik pembaca, sehingga terdapat zona buta antar \textit{checkpoint} \cite{hochreiter2024}. Akibatnya, estimasi posisi maupun kecepatan bergantung pada interpolasi sederhana. Selain itu, pemasangan infrastruktur RFID memerlukan biaya dan tenaga yang signifikan sehingga jumlah titik baca tidak dapat diperbanyak secara fleksibel, terutama pada rute maraton yang panjang.

\subsection{Model Sistem Berbasis LoRaWan dan GPS}

Sebagai perbaikan atas rendahnya resolusi spasial RFID, riset terbaru mengeksplorasi teknologi \textit{Low-Power Wide-Area Network} (LPWAN) seperti LoRaWAN untuk pelacakan kontinu. Riset mengusulkan model di mana pelari membawa pelacak GPS berbasis LoRa yang mengirimkan data posisi secara periodik, tidak terbatas pada titik diskrit\cite{hochreiter2024}.

Model konseptual dalam studi tersebut mencakup:
\begin{enumerate}
    \item \textit{Tracker Node:} Perangkat LoRa berbasis GPS yang mengirimkan pembaruan posisi secara berkala (misalnya setiap 30 detik).
    \item \textit{Mobile Gateway:} \textit{Gateway} bergerak yang dipasang pada kendaraan atau sepeda untuk meningkatkan cakupan sinyal di lingkungan lomba yang dinamis.
    \item \textit{LoRaWAN Network Server (LNS):} Server yang menerima, mendekode, dan menyimpan paket data untuk visualisasi dan analisis lebih lanjut.
\end{enumerate}

Eksperimen pada Vienna City Marathon menunjukkan bahwa pendekatan ini mampu menyediakan data posisi dengan resolusi jauh lebih tinggi dibandingkan RFID, yaitu median interval sekitar 31 detik. Namun, pendekatan ini menuntut infrastruktur \textit{backend} yang sanggup menangani aliran data telemetri yang terus-menerus dan berjumlah besar.

\subsection{Kesenjangan pada Metode Prediksi ETA}

Walaupun pelacakan kontinu semakin memungkinkan, metode prediksi ETA yang digunakan saat ini belum mampu memenuhi kebutuhan konteks lomba lari jarak jauh. Terdapat beberapa kesenjangan utama:

\begin{enumerate}
    \item Model yang Kurang Relevan: Banyak metode ETA yang tersedia dikembangkan untuk kendaraan atau logistik sehingga mengasumsikan kecepatan relatif stabil, bukan perubahan performa fisiologis pelari.
    \item Kurangnya Pertimbangan Konteks Lari: Metode konvensional sering kali tidak memperhitungkan karakteristik seperti elevasi rute, strategi \textit{pacing}, maupun dampak kelelahan.
    \item Kendala Skalabilitas dan Ketersediaan: Sistem ETA eksisting umumnya tidak dirancang untuk menghitung prediksi bagi ribuan peserta secara bersamaan setiap kali data baru diterima. Dalam \textit{event} berskala besar, diperlukan arsitektur \textit{backend} yang mampu menjamin \textit{high availability} dan pemrosesan \textit{real-time}.
\end{enumerate}

Dengan demikian, terdapat kebutuhan akan sistem \textit{backend} yang mampu mengolah data pelacakan kontinu sekaligus menyediakan prediksi ETA yang cerdas, skalabel, dan andal.


\section{Analisis Kebutuhan}
Tahap analisis kebutuhan dilakukan untuk memperoleh gambaran yang komprehensif mengenai tujuan, batasan, serta karakteristik sistem yang akan dibangun. Pada tahap ini, dilakukan identifikasi terhadap permasalahan dan kebutuhan pengguna yang muncul dalam operasional ITB Ultra-Marathon. Selain itu, dirumuskan kebutuhan fungsional yang mendefinisikan kemampuan utama yang harus disediakan sistem, serta kebutuhan nonfungsional yang memastikan sistem mampu beroperasi secara andal, aman, dan skalabel sesuai standar yang dibutuhkan. Hasil dari analisis kebutuhan ini menjadi dasar dalam proses perancangan arsitektur dan implementasi sistem pada tahap berikutnya.

\subsection{Identifikasi Masalah Pengguna}

Berdasarkan observasi dan wawancara mengenai kondisi penyelenggaraan ITB Ultra-Marathon saat ini, terdapat sejumlah permasalahan utama yang dialami oleh panitia, peserta, maupun pihak pendukung acara. Permasalahan tersebut dapat diidentifikasi sebagai berikut.

\begin{enumerate}
    \item Pelacakan posisi pelari masih manual dan tidak real-time.\\
    Panitia dan tim supporter harus meminta pelari melakukan \textit{share location} melalui aplikasi perpesanan untuk mengetahui posisi terkini, sehingga proses pemantauan menjadi tidak efisien.

    \item Mobil penjemput sering tidak tepat waktu karena ETA tidak akurat. \\
    Keterlambatan terjadi karena tidak tersedianya data posisi dan kecepatan pelari yang diperbarui secara kontinu untuk memperkirakan estimasi waktu kedatangan dengan tepat.

    \item Metode ETA yang tersedia (misalnya Google Maps) tidak sesuai untuk konteks pelari. \\
    Pola \textit{pace}, kelelahan, elevasi lintasan, serta karakteristik rute tidak diperhitungkan, sehingga estimasi waktu kedatangan sering meleset dan tidak dapat diandalkan.

    \item Peta digital lomba masih dibuat secara manual. \\
    Peta rute biasanya dibuat menggunakan Google Maps secara manual dan tidak terintegrasi dengan sistem pelacakan, meskipun pelari menggunakan aplikasi tersebut untuk mendapatkan arah rute dan estimasi waktu tempuh.

    \item Tidak ada sistem yang mendukung kebutuhan operasional lomba secara menyeluruh. \\
    Fitur-fitur penting seperti mode \textit{spectator} untuk supporter, manajemen tim, deteksi otomatis 16 segmen lomba, identifikasi pergantian pelari untuk kategori relay, serta pemantauan dan penegakan \textit{cut-off time} masih dilakukan secara manual dan rentan terhadap kesalahan.
\end{enumerate}


\subsection{Kebutuhan Fungsional}

\subsection{Kebutuhan Nonfungsional}
\lipsum[7]

\section{Analisis Pemilihan Solusi}
\subsection{Alternatif Solusi}
\lipsum[8]
\subsection{Analisis Penentuan Solusi}
\lipsum[9]