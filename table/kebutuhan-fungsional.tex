% functional requirement sistem
% \begin{center}
% \begin{longtable}{|c|p{3.5cm}|p{8.5cm}|}
% \caption{Kebutuhan Fungsional Sistem}\label{tab:kebutuhan-fungsional}\\
% \hline
% \textbf{Kode} & \textbf{Kebutuhan} & \textbf{Deskripsi} \\
% \hline
% \endfirsthead

% \hline
% \textbf{Kode} & \textbf{Kebutuhan} & \textbf{Deskripsi} \\
% \hline
% \endhead

% \endfoot

% \hline
% \endlastfoot

% FR-01 & Pendaftaran Peserta &
% Sistem menyediakan proses pendaftaran peserta, termasuk pengisian data identitas dan informasi dasar yang diperlukan untuk mengikuti lomba. \\
% \hline

% FR-02 & Manajemen Tim & Sistem memungkinkan peserta membuat dan bergabung ke tim, serta mengelola struktur tim yang beranggotakan pelari dan supporter. \\
% \hline

% FR-03 & Manajemen Lomba & Sistem mendukung pembuatan dan pengaturan lomba oleh panitia, termasuk jadwal, kategori, aturan, dan konfigurasi event. \\
% \hline

% FR-04 & Pengelolaan Trek & Sistem memungkinkan panitia membuat rute lomba, menetapkan lokasi checkpoint/water station, jam buka, dan cutoff time untuk setiap titik. \\
% \hline

% FR-05 & Pelacakan Peserta Real-time & Sistem menerima dan memproses data lokasi peserta secara kontinu untuk menyediakan posisi terbaru kepada pihak yang berhak. \\
% \hline

% FR-06 & Prediksi ETA & Sistem menghitung ETA peserta berdasarkan pace, progres rute, elevasi, dan dinamika performa pelari. \\
% \hline

% FR-07 & Deteksi Pergantian Pelari & Sistem mendeteksi pergantian pelari pada checkpoint dan mencatat perubahan pelari aktif dalam satu tim. \\
% \hline

% FR-08 & Pencatatan Status Peserta & Sistem mencatat status peserta (DNS, DNF, finished, disqualified, running) serta waktu start, checkpoint, penalti, dan waktu selesai. \\
% \hline

% FR-09 & Penjadwalan Penjemputan & Sistem menampilkan informasi penjemputan, layanan medis, dan drop bag dari panitia kepada peserta secara terorganisir. \\
% \hline

% FR-10 & Permintaan Bantuan & Sistem menyediakan fitur bagi peserta untuk mengirim permintaan pertolongan, berhenti, atau meminta evakuasi kepada panitia. \\
% \hline

% FR-11 & Manajemen Pengguna & Sistem menyediakan autentikasi dan otorisasi, termasuk pengaturan role (peserta, panitia, supporter) serta pembatasan akses sehingga pengguna hanya dapat melihat data sesuai perannya. \\
% \hline

% FR-12 & Notifikasi & Sistem memberikan notifikasi terkait kondisi lomba seperti nearing cutoff time, status darurat, atau perubahan status peserta. \\
% \hline

% FR-13 & Riwayat dan Audit Log & Sistem menyimpan riwayat posisi, kecepatan, status lomba, perubahan oleh panitia, serta log penting lainnya. \\
% \hline

% FR-14 & Dashboard Operasional & Sistem menyediakan dashboard data agregat untuk panitia, termasuk pelari mendekati cutoff dan statistik operasional. \\
% \hline

% FR-15 & Verifikasi Registrasi & Sistem mendukung verifikasi pembayaran atau kelengkapan administrasi sebelum peserta resmi terdaftar. \\
% \hline

% FR-16 & Sinkronisasi Offline & Sistem mendukung buffering data ketika koneksi buruk dan sinkronisasi ulang otomatis ketika koneksi pulih. \\
% \hline

% \end{longtable}
% \end{center}

% functional requirement backend
\begin{center}
\begin{longtable}{|c|p{3.5cm}|p{8.5cm}|}
\caption{Kebutuhan Fungsional Sistem}\label{tab:kebutuhan-fungsional}\\
\hline
\textbf{Kode} & \textbf{Kebutuhan} & \textbf{Deskripsi} \\
\hline
\endfirsthead

\hline
\textbf{Kode} & \textbf{Kebutuhan} & \textbf{Deskripsi} \\
\hline
\endhead

\endfoot

\hline
\endlastfoot

FR-01 & API Pendaftaran Peserta & 
Backend menyediakan API untuk pendaftaran peserta, termasuk validasi data identitas dan penyimpanan informasi dasar peserta ke dalam database. \\
\hline
FR-02 & API Manajemen Tim & 
Backend menyediakan endpoint untuk membuat, mengelola, dan menghubungkan peserta ke dalam tim, termasuk sinkronisasi struktur tim dengan modul lomba dan modul pelacakan. \\
\hline
FR-03 & API Manajemen Lomba & 
Backend menyediakan API bagi panitia untuk membuat, memperbarui, dan mengambil konfigurasi event seperti jadwal, kategori lomba, serta aturan dan parameter lomba. \\
\hline
FR-04 & API Pengelolaan Trek & 
Backend menyediakan API untuk menerima data rute lomba (GPX/TCX), pengaturan checkpoint dan water station, serta pendefinisian jam buka, jam tutup, dan cutoff time untuk setiap titik. \\
\hline
FR-05 & Akuisisi dan Distribusi Data Tracking & 
Backend menerima data lokasi peserta melalui endpoint ingestion dan menyediakan API streaming atau REST untuk menyalurkan data posisi terbaru kepada pihak yang berwenang. \\
\hline
FR-06 & Layanan Backend Perhitungan ETA & 
Backend menjalankan komponen perhitungan ETA berdasarkan pace, progres rute, elevasi, dan dinamika performa pelari, serta menyediakan API untuk pengambilan nilai ETA. \\
\hline
FR-07 & API Deteksi Pergantian Pelari & 
Backend menyediakan mekanisme serta API untuk mencatat dan mendeteksi pergantian pelari dalam satu tim berdasarkan checkpoint atau input panitia. \\
\hline
FR-08 & API Manajemen Status Peserta & 
Backend menyediakan endpoint untuk memperbarui status peserta (DNS, DNF, finished, disqualified, running), mencatat waktu start, checkpoint, penalti, dan waktu selesai. \\
\hline
FR-09 & API Permintaan Bantuan & 
Backend menyediakan endpoint bagi peserta untuk mengirim sinyal bantuan, permintaan berhenti, atau permintaan evakuasi, serta mendistribusikan event tersebut ke modul panitia. \\
\hline
FR-10 & Manajemen Pengguna dan RBAC & 
Backend menyediakan autentikasi, otorisasi, manajemen token, dan pembatasan akses sehingga pengguna hanya dapat melihat data sesuai perannya (panitia, peserta, supporter). \\
\hline
FR-11 & Penyimpanan Riwayat dan Audit Log & 
Backend menyimpan riwayat posisi, kecepatan, perubahan status, aktivitas panitia, serta log penting lainnya yang dapat diakses kembali melalui API. \\
\hline
FR-12 & Sinkronisasi dan Buffering Offline & 
Backend menyediakan mekanisme penerimaan data tertunda akibat koneksi buruk serta melakukan sinkronisasi ulang ketika koneksi pulih. \\
\hline

\end{longtable}
\end{center}
