\begin{center}
\begin{longtable}{|c|p{3.5cm}|p{8.5cm}|}
\caption{Kebutuhan Nonfungsional Sistem}\label{tab:kebutuhan-nonfungsional}\\
\hline
\textbf{Kode} & \textbf{Kebutuhan} & \textbf{Deskripsi} \\
\hline
\endfirsthead

\hline
\textbf{Kode} & \textbf{Kebutuhan} & \textbf{Deskripsi} \\
\hline
\endhead

\endfoot

\hline
\endlastfoot

NF-01 & Ketersediaan Sistem &
Sistem backend harus memiliki tingkat \textit{availability} minimal 99.9\% selama periode lomba, sehingga layanan inti seperti pelacakan real-time, ETA, dan dashboard panitia tetap dapat diakses tanpa gangguan. \\ \hline

NF-02 & Skalabilitas &
Sistem harus mampu menangani peningkatan jumlah pengguna, klien pelacakan, dan request API secara dinamis dengan mendukung mekanisme autoscaling dan arsitektur terdistribusi di lingkungan cloud. \\ \hline

NF-03 & Performa API &
Endpoint API harus merespons dalam waktu kurang dari 300 ms untuk operasi utama seperti pengambilan posisi pelari, status, dan data ETA, demi menjamin pengalaman real-time. \\ \hline

NF-04 & Konsistensi Data &
Sistem harus memastikan konsistensi data pelari, checkpoint, dan status lomba meskipun terdapat update simultan dari banyak perangkat dan panitia. Mekanisme seperti \textit{event-driven workflow} atau \textit{transactional safety} perlu diterapkan. \\ \hline

NF-05 & Keamanan &
Sistem harus menerapkan autentikasi dan otorisasi berbasis token, enkripsi data dalam transit (HTTPS/TLS), pembatasan akses berdasarkan peran, serta perlindungan terhadap akses tidak sah dan penyalahgunaan API. \\ \hline

NF-06 & Reliabilitas Telemetri &
Sistem harus mampu menerima data lokasi secara kontinu meskipun kondisi jaringan buruk melalui dukungan buffering, retry mechanism, atau strategi \textit{eventual delivery}. Data yang hilang harus diminimalkan. \\ \hline

NF-07 & Integritas ETA &
Model perhitungan ETA harus menghasilkan output yang stabil dan dapat diandalkan, serta tetap akurat meskipun data telemetri tidak lengkap atau mengalami delay. \\ \hline

NF-08 & Fault Tolerance &
Sistem harus tetap beroperasi meskipun terjadi kegagalan pada salah satu komponen (server, pod, database replica, dsb.) dengan menerapkan \textit{redundancy} dan \textit{graceful degradation}. \\ \hline

NF-09 & Monitoring dan Observability &
Sistem menyediakan logging terstruktur, metrik performa, tracing request, dan dashboard monitoring untuk mendeteksi anomali secara cepat. \\ \hline

NF-12 & Kapasitas Penyimpanan &
Sistem harus dapat menyimpan data telemetri dalam jumlah besar (posisi, status, event log) selama periode lomba tanpa mengorbankan performa query. Mekanisme TTL, cold storage, atau kompresi perlu didukung. \\ \hline

\end{longtable}
\end{center}
