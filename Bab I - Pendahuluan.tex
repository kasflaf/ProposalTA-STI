% ==========================================
% BAB I PENDAHULUAN
% ==========================================
\chapter{PENDAHULUAN}
\label{chap:pendahuluan}
% --- Latar Belakang ---
% NOTES: cek citation paragraf 1, tambahkan citation untuk paragraf lain
\section{Latar Belakang}

Ultra-Marathon merupakan salah satu cabang olahraga lari yang kini kian populer.
Hal ini ditunjukkan dengan kegiatan ITB Ultra-Marathon yang jumlah pesertanya terus meningkat sejak pertama kali diselenggarakan pada tahun 2017 \cite{hafizh2025}.
Seiring dengan meningkatnya partisipasi, kebutuhan akan sistem pelacakan pelari menjadi semakin penting.
Sistem tersebut berperan dalam memastikan keselamatan, performa, dan pengalaman peserta secara keseluruhan, serta memberikan interaktivitas bagi panitia dan pendukung acara \cite{hochreiter2024}.

Dalam penyelenggaraan ITB Ultra-Marathon saat ini, pelacakan posisi peserta masih dilakukan secara manual.
Kondisi ini menyulitkan panitia dalam memantau lokasi pelari secara \textit{real-time} dan menyebabkan koordinasi antarpos menjadi kurang efisien.
Dampaknya terlihat pada ketidakakuratan pengaturan mobil penjemput, yang kerap terlambat tiba di titik pengambilan peserta.
Ketiadaan sistem pemantauan khusus juga membuat panitia dan peserta mengandalkan aplikasi umum seperti Google Maps untuk memprediksi estimasi waktu kedatangan (ETA), meskipun aplikasi tersebut tidak dirancang untuk konteks pelari.
Permasalahan ini menegaskan perlunya mekanisme pelacakan yang terintegrasi, akurat, dan adaptif terhadap kebutuhan operasional marathon.

Rute marathon terdiri dari 16 segmen, dengan titik \textit{check point} (CP) dan \textit{water station} (WS) yang menjadi patokan untuk pemantauan \textit{real-time}.
Pelari dapat mengikuti kategori individu maupun relay, sehingga sistem harus mampu memantau pergantian anggota tim secara \textit{real-time} dan menampilkan status anggota aktif di setiap segmen.
Informasi posisi dan ETA pelari juga dapat dimanfaatkan panitia untuk mengatur transportasi dan shuttle support di jalur lomba.

Sistem komersial seperti MyLaps, ChronoTrack, dan Race Result menggunakan teknologi chip timing berbasis RFID untuk merekam waktu pelari secara otomatis di titik start, checkpoint, dan finish \cite{RaceID2022}.
Meskipun akurat dalam pencatatan waktu di titik-titik tertentu, sistem ini belum mendukung pemantauan posisi pelari secara kontinu di sepanjang lintasan.
Keterbatasan ini menghambat kemampuan penyelenggara untuk melakukan tracking secara \textit{real-time} dan memperkirakan ETA.

Sistem berbasis LoRaWAN dan GPS juga telah digunakan untuk pelacakan pelari dalam skala besar.
Pada uji coba dengan lebih dari 35.000 peserta, sistem ini mampu memperbarui posisi setiap 30 detik melalui \textit{mobile LoRaWAN gateway} pada kendaraan operasional, menunjukkan potensi untuk pelacakan \textit{real-time}.
Namun, LoRaWAN memerlukan arsitektur jaringan yang efisien dan terdistribusi untuk menjaga keandalan dan skalabilitas pengiriman data dalam sistem pelacakan berskala besar \cite{hochreiter2024}.

Mengenai prediksi ETA, metode yang tersedia saat ini umumnya dikembangkan untuk kendaraan, sehingga pendekatannya tidak sesuai untuk konteks pelari.
Sistem ETA tersebut tidak mempertimbangkan pola pace pelari, karakteristik rute, elevasi lintasan, maupun dinamika performa yang berubah sepanjang waktu.
Selain itu, metode ETA yang ada umumnya tidak dirancang untuk memproses data dari peserta dalam jumlah besar.

Keterbatasan berbagai solusi tersebut menunjukkan bahwa diperlukan sistem pelacakan yang dirancang khusus untuk kebutuhan ITB Marathon.
Oleh karena itu, penelitian ini mengusulkan perancangan dan implementasi \textit{backend} untuk \textit{Real-Time Runner Tracking} dan \textit{ETA Prediction} yang mampu beroperasi secara terukur (\textit{scalable}) serta memiliki tingkat ketersediaan layanan yang tinggi (\textit{high availability}).
Sistem ini diharapkan dapat menyediakan informasi posisi dan prediksi waktu kedatangan pelari secara lebih akurat, sekaligus mendukung proses operasional kepanitiaan secara efisien dan andal.

% --- Rumusan Masalah ---
\section{Rumusan Masalah}

Berdasarkan latar belakang yang telah diuraikan, rumusan masalah dalam penelitian ini adalah sebagai berikut.

\begin{enumerate}
    \item Bagaimana merancang dan mengimplementasikan sistem \textit{backend} untuk \textit{Real-Time Runner Tracking} yang mampu memantau posisi ribuan peserta ITB Marathon secara akurat dan \textit{real-time}, termasuk pemantauan per segmen, check point, dan status peserta (\textit{finisher} atau \textit{Did Not Finish})?
 
    \item Bagaimana merancang dan mengimplementasikan modul \textit{ETA Prediction} yang mampu menghasilkan estimasi waktu kedatangan pelari secara relevan dengan karakteristik pergerakan pelari dan kondisi rute?
 
    \item Bagaimana memastikan bahwa sistem \textit{backend} yang dibangun memiliki kemampuan skalabilitas (\textit{scalability}) dan ketersediaan layanan yang tinggi (\textit{high availability}) sehingga dapat beroperasi secara andal selama acara berlangsung?
 
    \item Bagaimana mengevaluasi kinerja sistem dalam menangani beban tinggi, khususnya dari aspek \textit{throughput}, \textit{latency}, \textit{error rate}, dan \textit{resource utilization}?
\end{enumerate}


% --- Tujuan ---
% NOTES: not sure with the criteria, does it have to be measureable? if so, need to add specific numbers/targets
\section{Tujuan}

Berdasarkan rumusan masalah yang telah disebutkan sebelumnya, tujuan dari penelitian ini adalah sebagai berikut.

\begin{enumerate}
    \item Menghasilkan rancangan dan implementasi sistem \textit{backend} untuk \textit{Real-Time Runner Tracking} yang mampu memantau posisi ribuan peserta ITB Marathon secara akurat dan \textit{real-time}, termasuk pemantauan per segmen, check point, dan status peserta (\textit{finisher}/DNF).
 
    \item Menghasilkan modul \textit{ETA Prediction} yang mampu memberikan estimasi waktu kedatangan pelari secara relevan dengan pola pergerakan pelari, kondisi rute, dan dinamika performa pelari.
 
    \item Menghasilkan arsitektur sistem \textit{backend} yang mampu mendukung skalabilitas (\textit{scalability}) dan ketersediaan layanan yang tinggi (\textit{high availability}) sehingga sistem tetap dapat beroperasi secara andal selama acara berlangsung.
 
    \item Menghasilkan evaluasi kinerja sistem dalam menangani beban tinggi, khususnya terkait \textit{throughput}, \textit{latency}, \textit{error rate}, dan \textit{resource utilization}.
\end{enumerate}

Kriteria keberhasilan dari penelitian ini ditetapkan sebagai berikut.

\begin{enumerate}
    \item Sistem \textit{Real-Time Runner Tracking} mampu memperbarui posisi pelari secara \textit{real-time} dengan tingkat keterlambatan pembaruan (update delay) yang berada dalam batas operasional yang dapat diterima oleh panitia.
 
    \item Modul \textit{ETA Prediction} mampu menghasilkan estimasi waktu kedatangan pelari dengan tingkat kesalahan prediksi yang rendah berdasarkan uji validasi pada data pergerakan pelari.
 
    \item Arsitektur sistem mampu menangani skala pengguna sesuai jumlah peserta ITB Marathon dan tetap beroperasi tanpa gangguan (\textit{downtime}) selama simulasi atau pengujian beban.
 
    \item Sistem memenuhi batas performa minimum pada pengujian beban, yang mencakup metrik \textit{throughput}, \textit{latency}, \textit{error rate}, dan \textit{resource utilization} sesuai target yang telah ditentukan.
 \end{enumerate}

% --- Batasan Masalah ---
% NOTES: sesuaikan beberapa poin dengan konisi asli
\section{Batasan Masalah}

Dalam penelitian ini, terdapat beberapa batasan yang digunakan untuk memfokuskan ruang lingkup pekerjaan dan memastikan hasil penelitian tetap relevan dengan tujuan yang telah ditetapkan. Batasan-batasan tersebut adalah sebagai berikut.

\begin{enumerate}
    \item Tugas akhir ini dikerjakan oleh dua orang mahasiswa, yaitu Dinda Thalia Fahira (18222055) dan Justin Lawrance (18222006), dengan pembagian fokus bahwa pengembangan \textit{backend} dilakukan oleh Justin Lawrance, sedangkan pengembangan \textit{frontend} dilakukan oleh Dinda Thalia Fahira.

    \item Implementasi yang dibahas pada laporan ini hanya mencakup pengembangan sistem \textit{backend} untuk \textit{Real-Time Runner Tracking} dan \textit{ETA Prediction}.  
    Pengembangan \textit{frontend} aplikasi tidak termasuk ruang lingkup pembahasan teknis pada laporan ini.

    \item Sistem \textit{backend} yang dikembangkan dibatasi pada fungsionalitas inti yang diperlukan untuk mendukung proses pelacakan dan prediksi, termasuk:  
    \begin{itemize}
        \item penerimaan dan pemrosesan data lokasi pelari.
        \item penyimpanan data posisi secara \textit{real-time}.
        \item penyediaan \textit{API} untuk konsumsi \textit{frontend}.
        \item modul prediksi waktu kedatangan (ETA).
        \item pemantauan status pelari per segmen dan Checkpoint/Water Station.
        \item deteksi \textit{off-route}.
    \end{itemize}

    \item Sistem hanya memproses rute lomba yang telah ditentukan panitia, termasuk segmentasi jalur dan lokasi Checkpoint/Water Station.
    Perubahan rute saat lomba tidak ditangani sistem.

    \item Backend mendukung kategori pelari individu dan tim relay dengan segmentasi tertentu.
    Pergantian pelari dicatat oleh sistem sesuai aturan panitia, namun status DNF atau penalti sepenuhnya ditentukan manual oleh panitia.

    \item Evaluasi nonfungsional difokuskan pada metrik \textit{performance} yang meliputi \textit{throughput}, \textit{latency}, \textit{error rate}, dan \textit{resource utilization}.  
    Evaluasi aspek lain seperti keamanan, biaya operasional, atau konsumsi energi tidak dibahas secara mendalam.

    \item Pengujian dilakukan menggunakan data simulasi yang merepresentasikan pergerakan pelari dalam skala besar.
    Pengujian tidak dilakukan dalam kondisi event sesungguhnya.

    \item Sistem tidak mencakup integrasi dengan perangkat pelacakan khusus (misalnya chip RFID atau sensor profesional) dan hanya memproses data lokasi berbasis koordinat yang dikirimkan dari aplikasi \textit{frontend}.
    
    \item Sistem tidak menangani kualifikasi dan pembayaran peserta. Akses terhadap fitur pendaftaran hanya diberikan oleh panitia kepada peserta yang terkualifikasi dan sudah melunasi pendaftaran.

    \item Sistem mencatat waktu, posisi, dan status peserta, namun tidak akan mengelola peringkat.
\end{enumerate}

% --- Metodologi Pengerjaan TA ---
\section{Metodologi}

Metodologi yang digunakan pada penelitian ini adalah pendekatan \textit{Waterfall Model} dari \textit{Software Development Life Cycle (SDLC)}.  
Pemilihan metodologi ini didasarkan pada batasan ruang lingkup sistem yang telah ditetapkan serta kebutuhan tahapan pengembangan yang sistematis dan terstruktur.  
Pengembangan dilakukan secara linear dan sekuensial, di mana setiap tahap harus diselesaikan sebelum melanjutkan ke tahap berikutnya, sehingga memudahkan pengukuran progres dan evaluasi hasil.
Berikut tahapan-tahapan yang dilakukan:

\begin{enumerate}
    \item Analisis Kebutuhan\\
    Pada tahap ini dilakukan pengumpulan dan analisis kebutuhan sistem backend untuk \textit{Real-Time Runner Tracking} dan \textit{ETA Prediction}.  
    Kegiatan dilakukan melalui studi literatur terkait sistem pelacakan, algoritma prediksi ETA, serta arsitektur backend yang scalable dan memiliki ketersediaan layanan yang tinggi.  
    Selain itu, dilakukan observasi dan wawancara dengan pengguna untuk memahami alur operasional, pembagian segmen lintasan, \textit{Water Station}, dan kebutuhan data posisi peserta.  
    Hasil dari tahap ini adalah spesifikasi kebutuhan fungsional dan non-fungsional sistem.

    \item Perancangan Sistem\\
    Berdasarkan hasil analisis kebutuhan, tahap perancangan mencakup penyusunan arsitektur sistem baik secara high-level maupun low-level, model database, spesifikasi API, \textit{sequence diagram}, serta arsitektur deployment yang dirancang untuk mendukung skalabilitas dan \textit{high availability}.
    Selain itu, perancangan juga mempertimbangkan rencana pengujian, pipeline deployment, dan mekanisme observability untuk memudahkan monitoring dan evaluasi sistem.

    \item Implementasi \\
    Tahap implementasi mencakup pengembangan backend sesuai desain yang telah dibuat.  
    Proses dimulai dengan setup lingkungan pengembangan dan infrastruktur, kemudian diikuti dengan implementasi modul penerimaan dan pemrosesan data posisi pelari secara \textit{real-time}, penyimpanan data, penyediaan API, serta perhitungan prediksi ETA.  
    Setiap modul dikembangkan secara sistematis agar backend berfungsi secara optimal dan sesuai spesifikasi.

    \item Deployment \\
    Tahap ini mencakup penyebaran sistem backend pada lingkungan \textit{cloud} 
    dengan arsitektur deployment yang mendukung (\textit{high availability}) dan skalabilitas.
    Konfigurasi sistem dirancang agar dapat berjalan stabil saat menerima beban ribuan peserta. 
    Tahap ini juga memastikan backend dapat diintegrasikan dengan frontend dan komponen lain untuk tahap pengujian.

    \item Pengujian \\
    Pengujian dilakukan untuk memastikan sistem berjalan sesuai spesifikasi.
    Tahap pengujian memiliki dua fokus utama.

    \begin{enumerate}
        \item Pengujian Fungsional \\
        Pengujian fungsional dilakukan untuk memastikan seluruh kebutuhan fungsional yang telah didefinisikan pada tahap perancangan berjalan sesuai spesifikasi. 
        Tahap ini mencakup pengecekan alur data, integrasi antar modul, validasi input dan output, serta mekanisme pelacakan dan perhitungan ETA yang menjadi inti dari sistem backend. 
            
        \item Pengujian Non-Fungsional \\
        Pengujian non-fungsional difokuskan pada evaluasi kinerja sistem dalam menangani beban tinggi dan menjaga kualitas layanan.  
        Metode yang digunakan meliputi load testing dan simulasi data ribuan peserta.  
        Metrik yang diukur mencakup \textit{throughput}, \textit{latency}, \textit{error rate}, dan \textit{resource utilization}.  
        Hasil pengujian didokumentasikan secara detail untuk mendukung evaluasi efektivitas dan skalabilitas sistem.
    \end{enumerate}

    \item Evaluasi \\
    Tahap evaluasi meliputi analisis hasil pengujian untuk menilai efektivitas sistem backend.  
    Analisis mencakup perbandingan performa prediksi ETA, akurasi data posisi pelari, dan pemenuhan kriteria keberhasilan yang telah ditetapkan.  
    Hasil evaluasi digunakan sebagai dasar dokumentasi temuan, pembelajaran, dan rekomendasi perbaikan sistem.

\end{enumerate}


