% ==========================================
% BAB I PENDAHULUAN
% ==========================================
\chapter{PENDAHULUAN}
\label{chap:pendahuluan}
% --- Latar Belakang ---
\section{Latar Belakang}
Ultra-Marathon merupakan salah satu cabang olahraga lari yang kini kian populer \cite{ronto2024}.
Hal ini ditunjukkan dengan kegiatan ITB Ultra-Marathon yang jumlah pesertanya terus meningkat sejak pertama kali diselenggarakan pada tahun 2017 \cite{hafizh2025}.
Seiring dengan meningkatnya partisipasi, kebutuhan akan sistem pelacakan pelari menjadi semakin penting.
Sistem tersebut berperan dalam memastikan keselamatan, performa, dan pengalaman peserta secara keseluruhan, serta memberikan interaktivitas bagi panitia dan pendukung acara \cite{hochreiter2024}.

Dalam penyelenggaraan ITB Ultra-Marathon saat ini, pelacakan posisi peserta masih dilakukan secara manual.
Kondisi ini menyulitkan panitia dalam memantau lokasi pelari secara real-time dan menyebabkan koordinasi antarpos menjadi kurang efisien.
Dampaknya terlihat pada ketidakakuratan pengaturan mobil penjemput, yang kerap terlambat tiba di titik pengambilan peserta.
Ketiadaan sistem pemantauan khusus juga membuat panitia dan peserta mengandalkan aplikasi umum seperti Google Maps untuk memprediksi estimasi waktu kedatangan (ETA), meskipun aplikasi tersebut tidak dirancang untuk konteks pelari.
Permasalahan ini menegaskan perlunya mekanisme pelacakan yang terintegrasi, akurat, dan adaptif terhadap kebutuhan operasional marathon.

Sejumlah solusi pelacakan posisi telah tersedia, seperti Google Maps Location Sharing, Glympse, atau fitur Live Location pada aplikasi \textit{instant messaging}.
Namun, layanan tersebut hanya ditujukan untuk pelacakan individu dan tidak dirancang untuk menangani pemantauan ribuan peserta secara simultan.
Selain itu, algoritma ETA pada layanan tersebut umumnya berorientasi pada pergerakan kendaraan sehingga tidak relevan untuk memodelkan perilaku pelari.

Di sisi lain, platform profesional seperti MYLAPS LiveTrack, RaceResult, dan Sporthive memang menawarkan solusi pelacakan untuk acara lari berskala besar.
Akan tetapi, sistem-sistem tersebut bersifat tertutup, memerlukan perangkat khusus, dan memiliki biaya operasional yang cukup tinggi, sehingga kurang sesuai dengan skala kegiatan ITB Marathon.
Selain itu, fleksibilitas untuk menyesuaikan sistem dengan alur kerja panitia sangat terbatas.
Oleh karena itu, meskipun solusi industri telah tersedia, tingkat keterbukaan, biaya, dan kemampuan integrasi menjadi hambatan utama dalam penerapannya.

Metode prediksi ETA yang ada saat ini juga belum mampu memenuhi kebutuhan event marathon secara akurat.
Layanan navigasi umum tidak mempertimbangkan pola pace pelari, kondisi rute, elevasi lintasan, serta dinamika performa dari waktu ke waktu.
Selain itu, belum tersedia solusi terbuka yang mampu memproses data ribuan pelari dan menghasilkan prediksi secara real-time dengan jaminan skalabilitas dan ketersediaan layanan yang tinggi.

Keterbatasan berbagai solusi tersebut menunjukkan bahwa diperlukan sistem pelacakan yang dirancang khusus untuk kebutuhan ITB Marathon.
Oleh karena itu, penelitian ini mengusulkan perancangan dan implementasi \textit{backend} untuk \textit{Real-Time Runner Tracking} dan \textit{ETA Prediction} yang mampu beroperasi secara terukur (\textit{scalable}) serta memiliki tingkat ketersediaan layanan yang tinggi (\textit{high availability}).
Sistem ini diharapkan dapat menyediakan informasi posisi dan prediksi waktu kedatangan pelari secara lebih akurat, sekaligus mendukung proses operasional kepanitiaan secara efisien dan andal.

% --- Rumusan Masalah ---
\section{Rumusan Masalah}
Rumusan Masalah berisi masalah utama yang dibahas dalam tugas akhir. Rumusan masalah yang baik memiliki struktur sebagai berikut:
\begin{enumerate}
\item	Pokok persoalan dari kondisi atau situasi yang ada saat ini. Dengan kata lain, jelaskan kelemahan atau kekurangan dari kondisi, situasi, atau solusi yang dijelaskan pada latar belakang. Ini merupakan pokok rumusan masalah.
\item	Elaborasi lebih lanjut urgensi penyelesaian masalah tersebut (mengapa penting untuk diselesaikan dan akibat yang dapat terjadi jika tidak diselesaikan).
\item	Usulan singkat terkait dengan solusi yang ditawarkan untuk menyelesaikan persoalan.
Penting untuk diperhatikan bahwa persoalan yang dideskripsikan pada subbab ini akan dipertanggungjawabkan di bab Evaluasi (apakah terselesaikan atau tidak).
\end{enumerate}

% --- Tujuan ---
\section{Tujuan}
Tuliskan tujuan utama dan/atau tujuan detail yang akan dicapai dalam pelaksanaan tugas akhir. Fokuskan pada hasil akhir yang ingin diperoleh setelah tugas akhir diselesaikan, terkait dengan penyelesaian persoalan pada rumusan masalah. Penting untuk diperhatikan bahwa tujuan yang dideskripsikan pada subbab ini akan dipertanggungjawabkan di akhir pelaksanaan tugas akhir apakah tercapai atau tidak. Tuliskan kriteria keberhasilan tugas akhir ini.

% --- Batasan Masalah ---
\section{Batasan Masalah}
Tuliskan batasan-batasan yang diambil dalam pelaksanaan tugas akhir. Batasan ini dapat dihindari (bersifat opsional, tidak perlu ada) jika topik atau judul tugas akhir dibuat cukup spesifik.

% --- Metodologi Pengerjaan TA ---
\section{Metodologi}
Tuliskan semua tahapan yang akan dilalui selama pelaksanaan tugas akhir. Tahapan ini spesifik untuk menyelesaikan persoalan tugas akhir. Khusus untuk penyusunan proposal ini, jelaskan secara detail:
\begin{enumerate}
\item	Tahapan investigasi pengumpulan fakta di latar belakang untuk merumuskan masalah.
\item	Langkah-langkah pencarian, pengelompokan, dan penapisan literatur atau sumber informasi untuk mengumpulkan informasi yang relevan tentang topik yang diangkat, termasuk teori (konsep atau teori apa saja yang perlu dicari), hal-hal yang telah dicapai oleh orang lain (cara mencari dan kata kuncinya), dan berbagai informasi pendukung, untuk mencari solusi terhadap masalah yang dibahas. Gunakan metodologi yang tepat dalam menggali informasi dan dokumentasikan prosesnya (termasuk rekaman wawancara atau survei) di dalam Lampiran, termasuk tautan ke video atau foto. Hasil penggalian informasi ini akan dijelaskan secara sistematis di Bab II Studi Literatur.
\end{enumerate}