% ==========================================
% BAB I PENDAHULUAN
% ==========================================
\chapter{PENDAHULUAN}
\label{chap:pendahuluan}
% --- Latar Belakang ---
% NOTES: cek citation paragraf 1, tambahkan citation untuk paragraf lain
\section{Latar Belakang}

Ultra-Marathon merupakan salah satu cabang olahraga lari yang kini kian populer \cite{ronto2024}.
Hal ini ditunjukkan dengan kegiatan ITB Ultra-Marathon yang jumlah pesertanya terus meningkat sejak pertama kali diselenggarakan pada tahun 2017 \cite{hafizh2025}.
Seiring dengan meningkatnya partisipasi, kebutuhan akan sistem pelacakan pelari menjadi semakin penting.
Sistem tersebut berperan dalam memastikan keselamatan, performa, dan pengalaman peserta secara keseluruhan, serta memberikan interaktivitas bagi panitia dan pendukung acara \cite{hochreiter2024}.

Dalam penyelenggaraan ITB Ultra-Marathon saat ini, pelacakan posisi peserta masih dilakukan secara manual.
Kondisi ini menyulitkan panitia dalam memantau lokasi pelari secara real-time dan menyebabkan koordinasi antarpos menjadi kurang efisien.
Dampaknya terlihat pada ketidakakuratan pengaturan mobil penjemput, yang kerap terlambat tiba di titik pengambilan peserta.
Ketiadaan sistem pemantauan khusus juga membuat panitia dan peserta mengandalkan aplikasi umum seperti Google Maps untuk memprediksi estimasi waktu kedatangan (ETA), meskipun aplikasi tersebut tidak dirancang untuk konteks pelari.
Permasalahan ini menegaskan perlunya mekanisme pelacakan yang terintegrasi, akurat, dan adaptif terhadap kebutuhan operasional marathon.

Rute marathon terdiri dari 16 segmen, dengan titik check point (CP) dan water station (WS) yang menjadi patokan untuk pemantauan real-time.
Selain memantau posisi pelari, sistem juga perlu mendukung penghitungan waktu terhadap batas maksimal (Cut-Off Time) di setiap CP/WS untuk membantu panitia menentukan status peserta, apakah finisher atau Did Not Finish (DNF).
Pelari dapat mengikuti kategori individu maupun relay, sehingga sistem harus mampu memantau pergantian anggota tim secara real-time dan menampilkan status anggota aktif di setiap segmen.
Informasi posisi dan ETA pelari juga dapat dimanfaatkan panitia untuk mengatur transportasi dan shuttle support di jalur lomba.

Sejumlah solusi pelacakan posisi telah tersedia, seperti Google Maps Location Sharing, Glympse, atau fitur Live Location pada aplikasi \textit{instant messaging}.
Namun, layanan tersebut hanya ditujukan untuk pelacakan individu dan tidak dirancang untuk menangani pemantauan ribuan peserta secara simultan.
Selain itu, algoritma ETA pada layanan tersebut umumnya berorientasi pada pergerakan kendaraan sehingga tidak relevan untuk memodelkan perilaku pelari.

Di sisi lain, platform profesional seperti MYLAPS LiveTrack, RaceResult, dan Sporthive memang menawarkan solusi pelacakan untuk acara lari berskala besar.
Akan tetapi, sistem-sistem tersebut bersifat tertutup, memerlukan perangkat khusus, dan memiliki biaya operasional yang cukup tinggi, sehingga kurang sesuai dengan skala kegiatan ITB Marathon.
Selain itu, fleksibilitas untuk menyesuaikan sistem dengan alur kerja panitia sangat terbatas.
Oleh karena itu, meskipun solusi industri telah tersedia, tingkat keterbukaan, biaya, dan kemampuan integrasi menjadi hambatan utama dalam penerapannya.

Metode prediksi ETA yang ada saat ini juga belum mampu memenuhi kebutuhan event marathon secara akurat.
Layanan navigasi umum tidak mempertimbangkan pola pace pelari, kondisi rute, elevasi lintasan, serta dinamika performa dari waktu ke waktu.
Selain itu, belum tersedia solusi terbuka yang mampu memproses data ribuan pelari dan menghasilkan prediksi secara real-time dengan jaminan skalabilitas dan ketersediaan layanan yang tinggi.

Keterbatasan berbagai solusi tersebut menunjukkan bahwa diperlukan sistem pelacakan yang dirancang khusus untuk kebutuhan ITB Marathon.
Oleh karena itu, penelitian ini mengusulkan perancangan dan implementasi \textit{backend} untuk \textit{Real-Time Runner Tracking} dan \textit{ETA Prediction} yang mampu beroperasi secara terukur (\textit{scalable}) serta memiliki tingkat ketersediaan layanan yang tinggi (\textit{high availability}).
Sistem ini diharapkan dapat menyediakan informasi posisi dan prediksi waktu kedatangan pelari secara lebih akurat, sekaligus mendukung proses operasional kepanitiaan secara efisien dan andal.

% --- Rumusan Masalah ---
\section{Rumusan Masalah}

Berdasarkan latar belakang yang telah diuraikan, rumusan masalah dalam penelitian ini adalah sebagai berikut.

\begin{enumerate}
    \item Bagaimana merancang dan mengimplementasikan sistem \textit{backend} untuk \textit{Real-Time Runner Tracking} yang mampu memantau posisi ribuan peserta ITB Marathon secara akurat dan real-time, termasuk pemantauan per segmen, check point, dan status peserta (\textit{finisher} atau \textit{Did Not Finish})?
 
    \item Bagaimana merancang dan mengimplementasikan modul \textit{ETA Prediction} yang mampu menghasilkan estimasi waktu kedatangan pelari secara relevan dengan karakteristik pergerakan pelari dan kondisi rute?
 
    \item Bagaimana memastikan bahwa sistem \textit{backend} yang dibangun memiliki kemampuan skalabilitas (\textit{scalability}) dan ketersediaan layanan yang tinggi (\textit{high availability}) sehingga dapat beroperasi secara andal selama acara berlangsung?
 
    \item Bagaimana mengevaluasi kinerja sistem dalam menangani beban tinggi, khususnya dari aspek \textit{throughput}, \textit{latency}, \textit{error rate}, dan \textit{resource utilization}?
\end{enumerate}


% --- Tujuan ---
% NOTES: not sure with the criteria, does it have to be measureable? if so, need to add specific numbers/targets
\section{Tujuan}

Berdasarkan rumusan masalah yang telah disebutkan sebelumnya, tujuan dari penelitian ini adalah sebagai berikut.

\begin{enumerate}
    \item Menghasilkan rancangan dan implementasi sistem \textit{backend} untuk \textit{Real-Time Runner Tracking} yang mampu memantau posisi ribuan peserta ITB Marathon secara akurat dan real-time, termasuk pemantauan per segmen, check point, dan status peserta (\textit{finisher}/DNF).
 
    \item Menghasilkan modul \textit{ETA Prediction} yang mampu memberikan estimasi waktu kedatangan pelari secara relevan dengan pola pergerakan pelari, kondisi rute, dan dinamika performa pelari.
 
    \item Menghasilkan arsitektur sistem \textit{backend} yang mampu mendukung skalabilitas (\textit{scalability}) dan ketersediaan layanan yang tinggi (\textit{high availability}) sehingga sistem tetap dapat beroperasi secara andal selama acara berlangsung.
 
    \item Menghasilkan evaluasi kinerja sistem dalam menangani beban tinggi, khususnya terkait \textit{throughput}, \textit{latency}, \textit{error rate}, dan \textit{resource utilization}.
\end{enumerate}

Kriteria keberhasilan dari penelitian ini ditetapkan sebagai berikut.

\begin{enumerate}
    \item Sistem \textit{Real-Time Runner Tracking} mampu memperbarui posisi pelari secara real-time dengan tingkat keterlambatan pembaruan (update delay) yang berada dalam batas operasional yang dapat diterima oleh panitia.
 
    \item Modul \textit{ETA Prediction} mampu menghasilkan estimasi waktu kedatangan pelari dengan tingkat kesalahan prediksi yang rendah berdasarkan uji validasi pada data pergerakan pelari.
 
    \item Arsitektur sistem mampu menangani skala pengguna sesuai jumlah peserta ITB Marathon dan tetap beroperasi tanpa gangguan (\textit{downtime}) selama simulasi atau pengujian beban.
 
    \item Sistem memenuhi batas performa minimum pada pengujian beban, yang mencakup metrik \textit{throughput}, \textit{latency}, \textit{error rate}, dan \textit{resource utilization} sesuai target yang telah ditentukan.
 \end{enumerate}

% --- Batasan Masalah ---
% NOTES: sesuaikan beberapa poin dengan konisi asli
\section{Batasan Masalah}

Dalam penelitian ini, terdapat beberapa batasan yang digunakan untuk memfokuskan ruang lingkup pekerjaan dan memastikan hasil penelitian tetap relevan dengan tujuan yang telah ditetapkan. Batasan-batasan tersebut adalah sebagai berikut.

\begin{enumerate}
    \item Tugas akhir ini dikerjakan oleh dua orang mahasiswa, yaitu Dinda Thalia Fahira (18222055) dan Justin Lawrance (18222006), dengan pembagian fokus bahwa pengembangan \textit{backend} dilakukan oleh Justin Lawrance, sedangkan pengembangan \textit{frontend} dilakukan oleh Dinda Thalia Fahira.

    \item Implementasi yang dibahas pada laporan ini hanya mencakup pengembangan sistem \textit{backend} untuk \textit{Real-Time Runner Tracking} dan \textit{ETA Prediction}.  
    Pengembangan \textit{frontend} aplikasi tidak termasuk ruang lingkup pembahasan teknis pada laporan ini.

    \item Sistem \textit{backend} yang dikembangkan dibatasi pada fungsionalitas inti yang diperlukan untuk mendukung proses pelacakan dan prediksi, termasuk:  
    \begin{itemize}
        \item penerimaan dan pemrosesan data lokasi pelari.
        \item penyimpanan data posisi secara real-time.
        \item penyediaan \textit{API} untuk konsumsi \textit{frontend}.
        \item modul prediksi waktu kedatangan (ETA).
        \item pemantauan status pelari per segmen dan Checkpoint/Water Station.
        \item deteksi \textit{off-route}.
    \end{itemize}

    \item Sistem hanya memproses rute lomba yang telah ditentukan panitia, termasuk segmentasi jalur dan lokasi Checkpoint/Water Station.
    Perubahan rute saat lomba tidak ditangani sistem.

    \item Backend mendukung kategori pelari individu dan tim relay dengan segmentasi tertentu.
    Pergantian pelari dicatat oleh sistem sesuai aturan panitia, namun status DNF atau penalti sepenuhnya ditentukan manual oleh panitia.

    \item Evaluasi nonfungsional difokuskan pada metrik \textit{performance} yang meliputi \textit{throughput}, \textit{latency}, \textit{error rate}, dan \textit{resource utilization}.  
    Evaluasi aspek lain seperti keamanan, biaya operasional, atau konsumsi energi tidak dibahas secara mendalam.

    \item Pengujian dilakukan menggunakan data simulasi yang merepresentasikan pergerakan pelari dalam skala besar.
    Pengujian tidak dilakukan dalam kondisi event sesungguhnya.

    \item Sistem tidak mencakup integrasi dengan perangkat pelacakan khusus (misalnya chip RFID atau sensor profesional) dan hanya memproses data lokasi berbasis koordinat yang dikirimkan dari aplikasi \textit{frontend}.
\end{enumerate}

% --- Metodologi Pengerjaan TA ---
\section{Metodologi}

Pelaksanaan tugas akhir ini mengikuti pendekatan \textit{Software Development Life Cycle (SDLC)} untuk memastikan pengembangan backend \textit{Real-Time Runner Tracking} dan \textit{ETA Prediction} dilakukan secara sistematis dan terdokumentasi dengan baik. Tahapan yang akan dilalui adalah sebagai berikut:

\begin{enumerate}
    \item Perencanaan\\
    Pada tahap ini dilakukan penentuan ruang lingkup, tujuan, batasan masalah, dan kriteria keberhasilan sistem.
    Kegiatan meliputi identifikasi permasalahan dalam penyelenggaraan ITB Ultra-Marathon, fokus pada pelacakan pelari dan prediksi ETA, serta perencanaan metodologi penelitian.

    \item Analisis Kebutuhan\\
    Tahap ini mencakup pengumpulan informasi terkait kebutuhan fungsional dan non-fungsional sistem backend.
    Metode yang digunakan meliputi studi literatur, observasi, wawancara, serta analisis data pendukung untuk menentukan modul-modul utama sistem, seperti pengolahan data posisi pelari, penyimpanan real-time, dan modul prediksi ETA.

    \item Perancangan Sistem\\
    Pada tahap perancangan dibuat arsitektur backend, diagram alur data, dan struktur penyimpanan informasi yang akan diimplementasikan.
    Desain ini bertujuan memudahkan proses pengembangan dan memastikan sistem dapat diukur skalabilitas dan ketersediaannya.

    \item Implementasi\\
    Tahap implementasi mencakup pengembangan modul backend sesuai desain yang telah disusun, termasuk penerimaan data lokasi pelari, pemrosesan posisi secara real-time, penyediaan API untuk frontend, dan perhitungan prediksi ETA.

    \item Pengujian\\
    Pengujian dilakukan untuk memastikan sistem berjalan sesuai spesifikasi, meliputi uji fungsional dan pengujian kinerja (\textit{throughput}, \textit{latency}, \textit{error rate}, dan \textit{resource utilization}).
    Pengujian menggunakan data simulasi yang menyerupai skala peserta ITB Ultra-Marathon.

    \item Deployment\\
    Tahap ini meliputi penyediaan sistem backend untuk demonstrasi prototipe, integrasi dengan frontend, serta penyusunan dokumentasi.
\end{enumerate}
